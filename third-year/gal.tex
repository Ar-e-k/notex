\documentclass[12pt, letterpaper]{article}

\usepackage{graphicx}
\usepackage{parskip} % Disabling paragraph index as it does not fit maths
\usepackage{hyperref} % Usable menu and references
\usepackage{amssymb} % Used to show sets of sumbers, like the real numbers

\graphicspath{{images}}

\title{Galois Theory}
\author{Arkadiusz Naks}
\date{2023}

\begin{document}

\begin{titlepage}
  \begin{center}
    \makeatletter
    \vspace*{1cm}
    \Huge
    \textbf{\@title}

    \vspace{0.5cm}
    \Large
    Lecture notes from Galois Theory at Durham University

    \vspace{1.5cm}

    \textbf{\@author}

    % \includegraphics[scale=0.55]{.png}
    \vfill

    \vspace{0.8cm}

    \small
    Based on my understanding of lectures and notes of \\
    \@date{}
  \end{center}
\end{titlepage}

\tableofcontents
\newpage

\begin{section}{Important Definisions}

  A place for short and important definisions \\

  \textsc{Definision} (Solutiob by Radicals) A solution that can be expressed by
  a finite number of basic arithmeic operations a well as \(\sqrt[m]{x}\)

\end{section}

\begin{section}{Introduction}

  The primary modivation of Galois Theory is to show that there is no general
  solution for polynomials of degree 5 or higher that can be \textbf{solved by
  radicals}. It can also be used to find can a particular given equasion can be
  \textbf{solved in radicals} as well as find the algorithm to find such
  solution (which in practice can be extremely complicated).

  \begin{subsection}{Cubics over \(\mathbb{C}\)}


    To find a general solution to an equation
    \(x^{3} + a_{2}x^{2} + a_{1}x + a_{0} = 0\) a substitution
    \(t = x + \frac{a_{2}}{3}\) can be used to eliminate the \(x^{2}\) term
    leading to \[t^{3} + pt + q\] where p and q can be expressed in terms of
    \(a_{2}, a_{1}, a_{0}\), namely \(p = \frac{-a^{2}_{2} + 3a_{1}}{3}\) and
    \(q = \frac{2a^{3}_{2} -9a_{1}a_{2} + 27a_{0}}{27}\). The new equation is
    called a \textbf{reduced} or \textbf{depressed} cubic. Next step is to do a
    substituton of \(p = -3uv\) and \(q = -(u^{3} + v^{3})\). This u and v also
    give a solution \(u + v\) to the \textbf{reduced} cubic. Starting from a
    quodriatic with roots \(u^{3}, v^{2}\), namely
    \((y - u^{3})(y - v^{3}) = 0\) it can be proceesed through the quodratic
    formula
    that \[u^{3}, v^{3} = -\frac{q}{2} \pm \sqrt{\frac{q^{2}}{4}} - \frac{p^{3}}{27}\],
    which helps optain a solution to the original equation by optaining the
    solutions in t, \(t = u + v\) and substituting back to x. The other two
    solutions can be obtained by multiplying either u or v by
    \(\omega = e^{\frac{2\pi i}{3}}\) and \(\omega^{2}\), (two other results for
    \(x^{3} = 1\)). The pairs of results are
    \(u + v, \omega u + \omega^{2} w, \omega^{2} u + \omega w\).

    The same formulas will give \textbf{solutions in radicals} in other fields
    \(\mathbb{F}_{p}\) for p prime and \(p \geq 5\). They fail for \(p = 2, 3\)
    as they involve divisibily by 2 and 3.

  \end{subsection}

  \begin{subsection}{Quortic over \(\mathbb{C}\)}

    Similarly to a cubic

  \end{subsection}

\end{section}

\end{document}
