\documentclass[12pt, letterpaper]{article}

\usepackage{graphicx}
\usepackage{parskip} % Disabling paragraph index as it does not fit maths
\usepackage{hyperref} % Usable menu and references
\usepackage{amssymb} % Used to show sets of sumbers, like the real numbers
\usepackage{mathrsfs} % Used for curly L and M

\newcommand{\Ll}{\mathscr{L}}
\newcommand{\Ls}{\(\Ll\)}

\newcommand{\p}{\textbf{P}}
\newcommand{\n}{\textbf{NP}}

\begin{document}

\begin{section}{Introduction}

  \begin{subsection}{Decision Problems}

    The standard definision of a decision problem is a problem that aims to
    anwser a yes/no question. Any decision problem can be identified with a
    language \(\sum\), with the question begin is a given string \(x \in
    \sum\). \\
    Decision problems can be classified as \textbf{decidable} and
    \textbf{undecidable}.

  \end{subsection}

  \begin{subsection}{Turing Machines}

    A k-tape Turing machine is a 5-touple \(\rangle Q, \sum, \delta, q_{0}, F
    \langle\), where
    \begin{itemize}
      \item Q is the finite set of states
      \item \(\sum\) is the finite alphabet
      \item \(\delta\) is the transition function
      \item \(q_{0} \in Q\) is the initial state
      \item \(F \subset Q\) is the set of finial states
    \end{itemize}

    Each tape consists a sequence of symbols for the langage \(lceil = \sum
    \cup \{ \triangle \}\) where \(\triangle\) is the blank symbol. \\
    The transition function is \(\delta : (Q / F) \times \lceil^{k} \to
    Q \times \lceil^{k} \times \{ L, S, R \}{}^{k}\) where L, S, K represent
    movement of each head over each tape.

    Time complexity of a turing machine T is a function (\(Time_{T}\)) s.t.\
    \(Time_{T}(x)\) is the number of steps T takes before terminating on input
    x. If \(T(x)\) does not halt, \(Time_{T}(x)\) is undefined.

    \begin{subsubsection}{Non Deterministic TMs}

      For NT a non deterministic Turing machine, then \(NT(x)\) is a tree of
      configurations which can be entered on an input x, and NT accepts x if at
      least one branch of the tree accepts x. Time complexity for such machine,
      denoted \(NTime_{NT}(x)\) is the number of steps in the longest branch of
      \(NT(x)\). If not all branches of \(NT(x)\) halt, NTime is undefined.

    \end{subsubsection}

  \end{subsection}

\end{section}

\begin{section}{Time Complexity}

  \begin{subsection}{Definision}

    For any language \Ls, it is said that \Ls{} is decidable in time
    \(O(f(x))\) if there exists a TM T which decides \Ls{} s.t.\
    \(Time_{T}(x) \leq cf(| x |)\) for all \(| x | > n_{0}\) where \(| x |\)
    denotes the lengh of x and c and \(n_{0}\) are numerical constants.

  \end{subsection}

  \begin{subsection}{Parameterized Problems}

    A parameterized problem is a problem with an input I that has distinguished
    k, called \textbf{parameter}. A \textbf{fixed-parameter tracable} (FTP) if
    it can be solved in \(f(k) n^{c}\), where c is a constant and \(f(k)\) is
    any function depending only on k.

    \begin{subsubsection}{Kernalization}

      This is a \textbf{polynomial-time} procedure that transforms \((I, k)\)
      to \((I', k')\) where \(k' \leq k\) and \(|I'| \leq f(k)\) for some
      function f. A problem having a \textbf{kernalizatoin} is equivelant to
      that problem begin FTP.

    \end{subsubsection}

  \end{subsection}

  \begin{subsection}{Complexity Classes}

    The time complexity class \(TIME[f]\) is the class of all languages with
    the time complexity \(O(f(x))\) (sometimes called DTIME for deterministic
    time). Some examples classes within decidable languages class are
    \[O(n) \subset O(n \log n) \subset O(n^{2}) \subset O(2^{n}).\]

    \begin{subsubsection}{P Class}

      The \p class is defined as \[\p = \cup_{k \geq 0} TIME[n^{k}].\] This
      class contains all languages which can be decided in a time bound by a
      finite polynomial. This class is \textbf{robust};
      meaning it does not depend on the computational model or the encoding.
      The best way to show a problem is in \p is to find the algorithm.

    \end{subsubsection}

    \begin{subsubsection}{NP Class}

      NP class can be defined similarly to the \p class, namely \[\n =
        \cup_{k \geq 0} NTIME[n^{k}].\] This makes the \n class the same thing
      the \p class is to Turing machines, just to non deterministic TMs.

      An alternative way to define \n class is a group of languages for which
      it can be verified that \(x \in \Ll\) given x in polynomial time. The
      class \p is a subset of \n.

      For \textbf{NP-completness}, a language \Ls{} needs all other languages
      in \n reduce to it via polynomial reduction. The first problem proven to
      be \textbf{NP-complete} is \textbf{satisfiability}.

    \end{subsubsection}

    \begin{subsubsection}{FTP Class}

      This is the class of all FTP problems. \(\p \subset FTP\). For a problem
      to be in FTP, any instance of this problem with fixed k must be in \p.
      This is a good way to prove problems are not in FTP. \\
      Similarly to \p, FTP is also \textbf{robust}.

    \end{subsubsection}

    \begin{subsubsection}{W[1] Class}

      Short Acceptance is a problem that takes a non-deterministic TM M, an
      input string x and a paramater k and checks is there a computation on M
      s.t.\ x accepts after at most k steps.

      This is a class of all problems which can be parameter reduced to Short
      Acceptance. This means Short Acceptance is \textbf{W[1]-complete}. \\
      \(FTP \subset W[1]\) \\
      As it is generally belived that \(W[1] \neq FTP\), if a problem is
      \textbf{W[1]-hard}, it is most likely not in FTP.

    \end{subsubsection}

  \end{subsection}

  \begin{subsection}{Reducibility}

    \begin{subsubsection}{Polynomial-Time}

      A language \(\Ll_{1}\) is \textbf{polynomially reducable} to \(\Ll_{2}\),
      denoted \(\Ll_{1} \leq \Ll_{2}\), if a polynomial time function \(f(x)\)
      exists s.t.\ \(x \in \Ll_{1} \iff f(x) \in \Ll_{2}\). If \(\Ll_{1} \leq
      \Ll_{2}\) and \(\Ll_{2} \in \p\) \(\Rightarrow \Ll_{1} \in \p\).

    \end{subsubsection}

    \begin{subsubsection}{Parameterized}

      A parameterized reduction exists from X to Y if \(\exists \phi\) s.t.\
      \begin{itemize}
        \item \(\phi(I)\) is YES instance of Y iff I is a YES instance of X
        \item \(\phi(I)\) can be computed in time \(f(k)| I |^{c}\)
        \item If k' is a parameter of \(\phi(I)\), \(k' \leq g(k)\) for some
              function g

      \end{itemize}

    \end{subsubsection}

  \end{subsection}

  \begin{subsection}{Completeness}

    A language \Ls{} is \textbf{complete} in its complexity class if it is the
    hardest problem in that class. This means every other problem in this class
    can be reduced to \Ls{} using a sufficient reduction algorithm. \Ls{} has
    to be in the class it is complete in.

    \begin{subsubsection}{Hardness}

      If a language is as hard as any other problem in a class but it is not in
      that class.

    \end{subsubsection}

  \end{subsection}

\end{section}

\end{document}
