\documentclass[12pt, letterpaper]{article}

\usepackage{graphicx}
\usepackage{parskip} % Disabling paragraph index as it does not fit maths
\usepackage{hyperref} % Usable menu and references
\usepackage{amssymb} % Used to show sets of numbers, like the real numbers
\usepackage{amsmath} % Used of matrix absolute value (vmatrix)
\usepackage{xargs} % Used for multiple deafult command values

\graphicspath{{images}}

\newcommand{\x}{\textbf{x}}
\newcommand{\norm}{\hat{\textbf{n}}}
\newcommand{\R}{\mathbb{R}}
\newcommandx{\scal}[2][1=f, 2=\x]{#1 (#2)}
\newcommandx{\vecs}[2][1=f, 2=\x]{\textbf{#1} (#2)}

\newcommand{\pder}[2]{\frac{\partial{} #1}{\partial{} #2}}
\newcommand{\der}[2]{\frac{d #1}{d #2}}

\title{Analysis in Many Variables}
\author{Arkadiusz Naks}
\date{2023}

\begin{document}

\begin{titlepage}
  \begin{center}
    \makeatletter
    \vspace*{1cm}
    \Huge
    \textbf{\@title}

    \vspace{0.5cm}
    \Large
    Lecture notes from AiMV at Durham University

    \vspace{1.5cm}

    \textbf{\@author}

    % \includegraphics[scale=0.55]{.png}
    \vfill

    \vspace{0.8cm}

    \small
    Based on my understanding of lectures and notes of \\
    \@date{}
  \end{center}
\end{titlepage}

\tableofcontents
\newpage

\begin{section}{Important Definisions}

  \textsc{Definision} (Curve) \textit{A funcion \(\R \Rightarrow \R^{3}\)}

  \textsc{Definision} (Surface) \textit{A funcion \(\R^{2} \Rightarrow \R^{3}\)
  }

  \textsc{Definision} (Scalar Field) \textit{A funcion \(\R^{n} \Rightarrow \R
    \)}

  \textsc{Definision} (Vector Field) \textit{A funcion \(\R^{n} \Rightarrow
    \R^{2}\)}

\end{section}

\begin{section}{Scaler Fields}

  For all this section let \(\scal = f(x, y, z)\).

  \begin{subsection}{Visualisation}

    A scalar field \(f(x, y)\)  in \(\R^{2}\) can be visualised by taking a
    constant height z, meaning drawing all points \((x, y)\) s.t.\ \(f(x, y)
    = z\). These are called \textbf{contours} of f. Similarly a scalar field in
    \(\R^{3}\) can be visualised as an \textbf{isosurfaces}, and
    computing those is a non trivial computation problem.

  \end{subsection}

  \begin{subsection}{Differentiation}

    Partial derivatives of f a n dimentional scalar field are encoded in its
    \textbf{gradient} vector which has the form \(\nabla f =
    (\pder{f}{x_{1}}, \dots , \pder{f}{x_{n}}){}^{T}\). To find the rate of
    change in a general direction with a unit vector \(\norm\) take the scalar
    product \(\norm \cdot \nabla f\). This is called a
    \textbf{directional derivative} of f. This is equivelant to \(| \nabla f |
    \cos(\theta)\), with \(\theta\) being the angle between \(\nabla f\) and
    \(\norm\).

    \(\nabla f\) is always orthogonal to the contours or isosurfaces in 2d and
    3d, or equivelants in higher dimentions.

  \end{subsection}

  \begin{subsection}{Curves}

    A curve C is a 1d subsection of \(\R^{3}\), usually described by a
    \textbf{parametrisation} \(\x(t) = (x(t), y(t), z(t))\) where \(t \in
    [a, b]\). The same curve can have different parametrisations going at
    different velocities. If a curve does not cross itself, it is called
    \textbf{simple} and if \(\x(a) = \textbf{x}(b)\) it is called
    \textbf{closed}.

    If the path is \textit{sufficiently smooth}, the parametrisation derivative
    can be definied similarly to the regular one. It is also called a
    \textbf{tangent curve}. If this derivative is non-zero everywhere, the
    parametrisation is called \textbf{regular}. For two different
    parametrisations \(\x_{1}(t_{1})\) and \(\textbf{x}_{2}(t_{2})\),
    then \(\der{\x_{1}}{t_{1}} = \der{\x_{2}}{t_{2}} \der{t_{2}}{t_{1}}\).
    There is a special parametrisation \(\x(s)\) called the \textbf{arclength}
    with \(\der{\x}{s} = 1 \; \forall s\) in the domain.

  \end{subsection}

  \begin{subsection}{Surfaces}

    There are 3 main ways to descrive a surface
    \begin{itemize}
      \item As a graph \(z = h(x, y)\) (\textbf{Explicit})
      \item As a level set of some scalar field \(\scal = 0\)
            (\textbf{Implicit})
      \item As \(\x(u, v) = (x(u, v), y(u, v), z(u, v))\)
            (\textbf{Parametricly})
    \end{itemize}

    At any point where the surface is \textit{sufficiently smooth}, a
    \textbf{unit normal} vector \(\norm \in \R^{3}\) is \textbf{orthogonal} to
    the surface. For an implicit surface \(\norm = \pm
    \frac{\nabla f}{| \nabla f |}\). For the explicit case, set \(\scal =
    h(x, y) - z\) and proceede same as for the implicit. \\
    For the parametric case \(\x(u, v)\), \[\norm = \frac{\pder{\x}{u} \times
        \pder{\x}{v}}{ | \pder{\x}{u} \times \pder{\x}{v} | }.\]
    This \(\norm\) depends on the ordering of u and v.

    A surface is called \textbf{smooth} if the \(\norm\) varies continously
    over the surface and \textbf{piecewise smooth} if it can be divided into
    finitely many \textbf{smooth} portions. It is \textbf{orientable} if the
    unit normal is consistant (a non orientable example is a Mobius strip).
    A surface can be \textbf{open} or \textbf{closed} depending if it can or
    cannot be bounded by the curve. For an open surface S, the boundry is
    denoted as \(\partial S\).

  \end{subsection}

  \begin{subsection}{Integrals}

    \begin{subsubsection}{Line Integrals}

      For C a regular arc of simple parametrisation \(\x(s) =
      \x\) with \(s \in [0, l]\), a \textbf{line integral} of \(f\) along
      C is \[\int_{C} \scal ds = \int^{l}_{0} f(x(s), y(s), z(s)) ds.\] This can
      be calculated for any other parametrisation \(\x(t)\) with
      \(t \in [a, b]\) as \(\int_{C} \scal = \int^{b}_{a} \scal
      | \der{\x}{t} | dt\). The result does not depend on the
      parametrisation. If a curve C is composed of multiple arcs, then the
      integral is the sum of them all. For a closed arc, the integral is denoted
      \(\oint\).

    \end{subsubsection}

    \begin{subsubsection}{Surface Integrals}

      For \(\scal\) a scalar field and \(\x(u, v)\) a parametrisation for a
      surface S, a surface integral is defined as \[\int_{S} \scal dS =
        \int_{U} \x(u, v) | \pder{\x}{u} \times \pder{\x}{v} | dudv\]
      where U is the region of \(u, v\).

    \end{subsubsection}

    \begin{subsubsection}{Volume Integrals}

      For \(\scal\) a scalar field and \(V \subset \R^{3}\) a volume integral
      is defined as \[\int_{V} \scal dV = \int_{x} \int_{y} \int_{z} \scal
        dzdydx.\]

    \end{subsubsection}

  \end{subsection}

\end{section}

% Add index notation later

\begin{section}{Vector Fields}

  A vector field in 3D is \(\vecs = (f(x), f(y), f(z)){}^{T}\).

  \begin{subsection}{Integrals}

    \begin{subsubsection}{Line Integrals}

      For \(\vecs\) a vector field and C an arc of a simple curve with
      parametrisation \(\x = \x(t)\) and \(t \in [0, l]\)
      \[\int_{C} \vecs d \x = \int_{C} \vecs \der{\x}{t} dt.\]

    \end{subsubsection}

    \begin{subsubsection}{Conservative}

      \(\vecs\) is called \textbf{conservative} if \(\vecs = \nabla \scal[g]\)
      for \(\scal[g]\) a scalar field called a \textbf{potential}. Then for a
      curve C between \(\x = \textbf{a}\) and \(\x = \textbf{b}\) the
      \textbf{line integral} of the vector field \(vecs\) can be calculated byy
      \[\int_{C} \vecs d \x = \int_{C} \nabla \scal[g] d \x = g(\textbf{a} -
        g(\textbf{b})).\]
      This implies that any curve between the same two points have same line
      integral. This is true only for \textbf{conservative} vecotr fields.

    \end{subsubsection}

    \begin{subsubsection}{Circulation}

      For a \textbf{closed} curve C, the line integral of \(\vecs\) is written
      \(\oint_{C} \vecs d \x\) and is called the \textbf{circulation} of
      \(vecs\) around C. If possible, C is traversed anticlockwise. Any
      circulation on a \textbf{conservative} vector field is 0. \\
      Otherwise for \(A \subset \R^{2}\) a bounded region with \(\partial A =
      C\) is piecewise smooth in the xy-plane, then if C is traversed with A on
      the left
      \[\oint_{C} \vecs d \x = \int_{A} (\pder{f_{2}}{x} - \pder{f_{1}}{y})
        dA\]
      where \(f_{1}, f_{2}\) are the first two components of the vector field.


    \end{subsubsection}

    \begin{subsubsection}{Surface Integrals}

      For S an \textbf{orientable} surface, \textbf{flux} of \(\vecs\) is
      defined as the surface integral \[\int_{S} \vecs d \textbf{S} =
        \int_{S} \vecs \norm dS.\] This is equivelant to saying
      \(d \textbf{S} = \norm dS\). The sign of the integral depends on the
      chosen \(\norm\). \\
      If the surface is closed, the integral is ofter writen as \(\oint\) and
      the convention is for \(\norm\) to point outwards.

    \end{subsubsection}

  \end{subsection}

  \begin{subsection}{Curl}

    The \textbf{curl} of \(\vecs\) is a vector field \(\nabla \times \vecs\)
    whose component in the \(norm\) direction is \[(\nabla \times \vecs) \norm{}
      = \lim_{| A \to{} 0|} \frac{1}{| A |} \oint_{C} \vecs d \x\] where C is
    in the plane normal to \(\norm\) and \(| A |\) is its area. This is
    calculated right handed relative to the direction of \(\norm\).
    \[\nabla \times \vecs = \begin{vmatrix}
                                e_{1} & e_{2} & e_{3} \\
                                \pder{}{x} & \pder{}{y} & \pder{}{z} \\
                                f_{1} & f_{2} & f_{3} \end{vmatrix}\]
    where \(e_{1}, e_{2}, e_{3}\) are unit vectors in the base 3 directoins.

    For vector fields \(\vecs, \vecs[g]\) and a scalar field \(\scal[h]\), as
    well as scalars \(a, b\)
    \begin{itemize}
      \item \(\nabla \times (a \vecs + b \vecs[g]) =
            a \nabla \times \vecs + b \nabla \times \vecs[g]\)
      \item \(\nabla \times (\scal[h] \vecs) =
            (\nabla \scal[h]) \times \vecs + \scal[h] (\nabla \times \vecs)\)
      \item \(\nabla (\nabla \scal[h]) = \textbf{0}\)
    \end{itemize}

  \end{subsection}

  \begin{subsection}{Divergance}

    \textbf{Divergance} of \(\vecs\) is a scalar field \[\scal[g] =
      \nabla \cdot \vecs = \lim_{| V | \to 0} \frac{1}{| V |} \oint_{S} \vecs d
      \textbf{S}\] where \textbf{S} is a closed surface enclosing the volume V.
    \\
    This represents \textbf{flux} per unit volume away from a given point.

    \(\nabla \cdot \vecs = \pder{f_{1}}{x} + \pder{f_{2}}{y} + \pder{f_{3}}{z}\)

    When \(\nabla \cdot \vecs > 0\), the point is a source (vector arrows go
    away) while for \(\nabla \cdot \vecs < 0\) it is a sink (vector arrows go
    towards it). Lastly for \(\nabla \cdot \vecs = 0\), there is no pattern.

    For vector fields \(\vecs, \vecs[g]\) and a scalar field \(\scal[h]\), as
    well as scalars \(a, b\)
    \begin{itemize}
      \item \(\nabla \cdot (a \vecs + b \vecs[g]) =
            a \nabla \cdot \vecs + b \nabla \cdot \vecs[g]\)
      \item \(\nabla \cdot (\scal[h] \vecs) =
            (\nabla \scal[h]) \cdot \vecs + \scal[h] (\nabla \cdot \vecs)\)
      \item \(\nabla \cdot (\nabla \times \vecs) = \textbf{0}\)
    \end{itemize}

    These are very similar to the properties of curl.

  \end{subsection}

\end{section}

\end{document}
