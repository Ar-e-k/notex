\documentclass[12pt, letterpaper]{article}

\usepackage{graphicx}
\usepackage{parskip} % Disabling paragraph index as it does not fit maths
\usepackage{hyperref} % Usable menu and references
\usepackage{amssymb} % Used to show sets of sumbers, like the real numbers

\graphicspath{{images}}

\title{Cryptography}
\author{Arkadiusz Naks}
\date{2023}

\begin{document}

\begin{titlepage}
  \begin{center}
    \makeatletter
    \vspace*{1cm}
    \Huge
    \textbf{\@title}

    \vspace{0.5cm}
    \Large
    Lecture notes from Cryptography half of Cryptography and Codes module at Durham University

    \vspace{1.5cm}

    \textbf{\@author}

    % \includegraphics[scale=0.55]{.png}
    \vfill

    \vspace{0.8cm}

    \small
    Based on my understanding of lectures and notes of \\
    \@date{}
  \end{center}
\end{titlepage}

\tableofcontents
\newpage

\begin{section}{Important Definisions}

  A place for short and important definisions \\

  \textsc{Definision} (Key Size) \textit{It is defined to be \(\log_{2}(x)\)
    where x is the number of all possible keys}

\end{section}

\begin{section}{Aims}

  There are multiple reasons of varying importance for encrypting comunication
  \begin{itemize}
    \item \textbf{Secrecy}: Hide content from outside viewers
    \item \textbf{Integrity}: Ensure the message cannot be tampered with
          (without the receiving side knowing)
    \item \textbf{Authenticity}: Similar to \textbf{integrity}, ensure that the
          receiver knows the message comes from where it claims it does
    \item \textbf{Non-Repudiation}: Make sure the sender cannot claim they did
          notes send the message
  \end{itemize}

  \textbf{Kerchhoffs's principle}: It is important cryptographic systems are
  still secure iven if the enemy knows the details of their method.
  There are multiple methods of attack, with the main catigories being:
  \begin{itemize}
    \item \textbf{Ciphertext-only}: Enemy obtains ciphertext and deduces the
          plaintext. Mostly relevant in naive systems
    \item \textbf{Know plaintext}: Enemy knows some plaintext and ciphertext and
          uses it to deduce the key
    \item \textbf{Chosen plaintext}: Enemy tricks the sender to encode some
          chosen planetext and observes the resultant cyphertext (similar to
          \textbf{Known plaintext})
    \item \textbf{Chosen ciphertext}: Enemy trickes the receiver into decoding
          some chosen ciphertexts and observers the result (\textbf{Chosent
          plaintext})
    \end{itemize}

    Types of attacks:
    \begin{itemize}
      \item \textbf{Brute force} Try to guess the key at random, can be simply
            avoided by having a large key size
      \item There is an attack on RSA encryption based on how long a machine
            takes to decode a given message
    \end{itemize}

\end{section}

\begin{section}{Symetric Key Ciphers}

  \begin{subsection}{Introduction}

    Symetric key ciphers use the same key for encrypting and decrypting data.
    This introduces a huge problem in sharing the key to the recipiant of the
    message, making this method largely outdated for usage in overinternet
    comunications.

  \end{subsection}

  \begin{subsection}{Substitution ciphers}

    Caesar cipher is one example of such ciphers. In this ciphers each letter is
    assigned another letter.

    Substiturion ciphers are very volnerable to \textbf{known plaintext} attacks
    as the key can be very easly build from there. It is also quite volnerable
    to \textbf{ciphertext-only} attacks if enough ciphertext is knonw. This is
    due to its regularity the statistics about letters and their pairing in the
    given language can be exploited to deduce the key.

  \end{subsection}

  \begin{subsection}{One time pads}

    In terms of algorithmic security, this cipher is perfect. Each letter
    \(p_{i}\) in the message is shifter by a number \(k_{i}\) which is on a key.
    More formally the key is \(k_{1}, k_{2}, \dots\) with each k being an
    independent random number from \(\mathbb{C}/26\) (or different if different
    alphabet is used). This means that for a plaintext \(p_{1}p_{2} \dots\) and
    ciphertext \(c_{1}c_{2} \dots\) each \(c_{i} = p_{i} + k_{i}\). This is
    provably secure, the distribution of the plaintext letters is independent of
    that of the ciphertext letters, so the ciphertext conveys no information
    about the plaintext.

    The disadvantage of this method is that the key size has to be enormous as
    it has to be at least as long as the message that is intended to be send (or
    as long as all messages to be send if its to be used for a full
    conversation). This introduces a massive problem with distributing the key
    which makes this method unviable in todays world communication. It was
    historically most notably used by the military with keys stored in books.

    It is important that the number is truly random for the method to be secure,
    otherwise the method of key generation can be deduced and key can be broken.

  \end{subsection}

\end{section}

\end{document}
