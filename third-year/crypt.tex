\documentclass[12pt, letterpaper]{article}

\usepackage{graphicx}
\usepackage{parskip} % Disabling paragraph index as it does not fit maths
\usepackage{hyperref} % Usable menu and references
\usepackage{amssymb} % Used to show sets of sumbers, like the real numbers

\graphicspath{{images}}

\title{Cryptography and Codes}
\author{Arkadiusz Naks}
\date{2023}

\begin{document}

\begin{titlepage}
  \begin{center}
    \makeatletter
    \vspace*{1cm}
    \Huge
    \textbf{\@title}

    \vspace{0.5cm}
    \Large
    Lecture notes from Cryptography and Codes module at Durham University

    \vspace{1.5cm}

    \textbf{\@author}

    % \includegraphics[scale=0.55]{.png}
    \vfill

    \vspace{0.8cm}

    \small
    Based on my understanding of lectures and notes of \\
    \@date{}
  \end{center}
\end{titlepage}

\tableofcontents

\newpage

\begin{section}{Important Definisions}

  A place for short and important definisions

  \textsc{Definision} (Key Size) \textit{It is defined to be \(\log_{2}(x)\)
    where x is the number of all possible keys}

  \textsc{Definision} (Confusion) \textit{Each symbol in the ciphertext depends
    on many parts of the key}

  \textsc{Definision} (Diffusion) \textit{Each symbol in the ciphertext depends
    on many parts of the plaintext}

  \textsc{Definision} (\(\mathbb{Z}/n \mathbb{Z}^{x}\)) \textit{Set of all
    numbers m s.t.\ \(\gcd(n, m) = 1, \varphi(n) = amount\) of elements in this
    set}

\end{section}

\begin{section}{Cryptography Aims}

  There are multiple reasons of varying importance for encrypting comunication
  \begin{itemize}
    \item \textbf{Secrecy}: Hide content from outside viewers
    \item \textbf{Integrity}: Ensure the message cannot be tampered with
          (without the receiving side knowing)
    \item \textbf{Authenticity}: Similar to \textbf{integrity}, ensure that the
          receiver knows the message comes from where it claims it does
    \item \textbf{Non-Repudiation}: Make sure the sender cannot claim they did
          notes send the message
  \end{itemize}

  \textbf{Kerchhoffs's principle}: It is important cryptographic systems are
  still secure iven if the enemy knows the details of their method.
  There are multiple methods of attack, with the main catigories being:
  \begin{itemize}
    \item \textbf{Ciphertext-only}: Enemy obtains ciphertext and deduces the
          plaintext. Mostly relevant in naive systems
    \item \textbf{Know plaintext}: Enemy knows some plaintext and ciphertext and
          uses it to deduce the key
    \item \textbf{Chosen plaintext}: Enemy tricks the sender to encode some
          chosen planetext and observes the resultant cyphertext (similar to
          \textbf{Known plaintext})
    \item \textbf{Chosen ciphertext}: Enemy trickes the receiver into decoding
          some chosen ciphertexts and observers the result (\textbf{Chosent
          plaintext})
    \end{itemize}

    Types of attacks:
    \begin{itemize}
      \item \textbf{Brute force} Try to guess the key at random, can be simply
            avoided by having a large key size
      \item There is an attack on RSA encryption based on how long a machine
            takes to decode a given message
    \end{itemize}

\end{section}

\begin{section}{Symmetric Key Ciphers}

  \begin{subsection}{Introduction}

    Symmetric key ciphers use the same key for encrypting and decrypting data.
    This introduces a huge problem in sharing the key to the recipiant of the
    message, making this method largely outdated for usage in overinternet
    comunications.

  \end{subsection}

  \begin{subsection}{Substitution ciphers}

    Caesar cipher is one example of such ciphers. In this ciphers each letter is
    assigned another letter.

    Substiturion ciphers are very volnerable to \textbf{known plaintext} attacks
    as the key can be very easly build from there. It is also quite volnerable
    to \textbf{ciphertext-only} attacks if enough ciphertext is knonw. This is
    due to its regularity the statistics about letters and their pairing in the
    given language can be exploited to deduce the key. Such an attack is know
    as a \textbf{frequency attack}.

  \end{subsection}

  \begin{subsection}{One time pads}

    In terms of algorithmic security, this cipher is perfect. Each letter
    \(p_{i}\) in the message is shifter by a number \(k_{i}\) which is on a key.
    More formally the key is \(k_{1}, k_{2}, \dots\) with each k being an
    independent random number from \(\mathbb{C}/26\) (or different if different
    alphabet is used). This means that for a plaintext \(p_{1}p_{2} \dots\) and
    ciphertext \(c_{1}c_{2} \dots\) each \(c_{i} = p_{i} + k_{i}\). This is
    provably secure, the distribution of the plaintext letters is independent of
    that of the ciphertext letters, so the ciphertext conveys no information
    about the plaintext.

    The disadvantage of this method is that the key size has to be enormous as
    it has to be at least as long as the message that is intended to be send (or
    as long as all messages to be send if its to be used for a full
    conversation). This introduces a massive problem with distributing the key
    which makes this method unviable in todays world communication. It was
    historically most notably used by the military with keys stored in books.

    It is important that the number is truly random for the method to be secure,
    otherwise the method of key geration can be deduced and key can be broken.

  \end{subsection}

  \begin{subsection}{Hill cipher}

    The plaintext is dived into locks of length n, with each block represented
    as a vector. The key is an inverteble \(n \times n\) matrix. The
    ciphertext is obtained by \(C_{i} = MP_{i}\) where M is the key matrix and
    \(P_{i}\) are the plaintext blocks.

    The \(i^{th}\) character in the ciphertext depends on \(i^{th}\) row of
    the key (\textbf{confusion}). \\
    The \(i^{th}\) character in the ciphertext depends on the \(j^{th}\)
    character in the plaintext if the \(M_{ij} \neq 0\) (\textbf{diffusion}).

    Not very good against \textbf{known plaintext} attacks.

  \end{subsection}

\end{section}

\newpage

\begin{section}{Public Key Cryptography}

  In a public key cryptosystems encipherigng and deciphering are done by
  distinct keys D and E, s.t.\ computing D from E is computationaly not fusable.
  \(k_{E}\) is send out to the public to encrypt the messages to the receiver
  and \(k_{D}\) is private and used to decode.

  \begin{subsection}{RSA}

    \(k_{E} = (n, e)\) where \(n = pq\) with p and q prime and e is an
    integer. \(k_{D} = d\) s.t.\ \(de \equiv 1 \pmod{\varphi(n)}\). The
    message m is an integer modulo n and c is the code. \\
    Encription: \[c \equiv m^{e} \pmod{n}\]
    Decription: \[m \equiv c^{d} \pmod{n}\]

    Typical \(e = 65537\) which is \(2^{16} + 1\). More generally, e has to be
    chosen so \(\gcd(e, (p - 1)(q - 1)) = \gcd(e, \varphi(n)) = 1\). This is
    best ensured by making e prime.

    \begin{subsubsection}{Signature}

      RSA can also be used for message \textbf{authentication}. This is achived
      by a \textbf{signature} \(s \equiv m^{d} \pmod{n}\). To verify the
      signature, the recipiant can perform \(s^{e} \equiv m \pmod{n}\). This is as
      hard to break as the usual RSA problem discussed above. This is not perfect
      as an attacker can chose a signature first and then construct the message
      based on it, producing a most likely nonsense but validaly signed message. \\
      This can be fixed by introducing a \textbf{hash function} h so that
      \(s = h(m){}^{d}\) (where the hash function is public). In that case the
      recipiant checks if \(h(m) \equiv s^{e} \pmod{n}\). In this case to do the
      same attack, the inverse of the \textbf{hash} has to be found, which is
      defined to be a hard problem (the function is chosen so it is).

    \end{subsubsection}

    \begin{subsubsection}{Security}

      Security of RSA depends on how hard it is to find m given n and e. This
      problem is equivelant to finding the modular \(e^{th}\) root of c. This
      largely depends on the problem of factorising n into p and q qith only
      the knowladge of n. \\
      \textit{Whenether the converse is true is still an open problem (RSA
        Problem)}.

      It is no easier to find \(\varphi(n)\) than it is to factorise n, as well
      as it is no easier to find d than it is to factorise n. This means that
      breaking RSA is as hard as factorising n.

      If either one of p, q or d are found, it is easy to break the cipher.

      The primes p and q have to be secure against attacks. If they are too
      small, an algorithm which simply tests all primes from lowest to highest
      can factorise n very quickly (\textbf{trial division}). On the other hand
      if p and q are too close \(|p - q| \leq n^{\frac{1}{4}}\),
      \textbf{fermat's method} can be used to f ind them in a resonable time.

      \textbf{Fermat's method}: \\
      Set \(x = \lceil \sqrt{n} \rceil\) then check if \(x^{2} - n = y^{2}\)
      for any integer y. If not increase x by 1 and check again. If it is,
      it implies that \(x^{2} \equiv y^{2} \pmod{n} \Rightarrow x - y | p\)
      (or q). It is important that \(x^{2} \not\equiv \pm y^{2} \pmod{n}\).

    \end{subsubsection}

  \end{subsection}

  \begin{subsection}{Hash function}

    Defined as a function that
    \begin{itemize}
      \item Efficient
      \item Preimage-resistant: cannot quickly find m given \(h(m)\)
            (hard to find inverse)
      \item Collision-resistant: cannot quickly find m, m' s.t.\
            \(h(m) = h(m')\)
    \end{itemize}

    sha256 is a commanly used hash function.

  \end{subsection}

  \begin{subsection}{Discrete Logarithms}

    Recall that for a prime p \((F_{p}, \times)\) is a multiplicative group
    (denoted as \(F_{p}^{\times}\)) generated by an element \(g \in F_{p}\),
    for g a \textbf{primitive root} mod p. This also implies \(ord(g) = p - 1\)
    mod p.

    \textit{Whenether there are infinetly many ps s.t.\ 2 is a primitive root
      mod p is an open question in maths, but generally it is considered true
      and can be proven with the general version of the Reimann hypothesis.}

    Take p prime and \(g \in F_{p}^{\times}\) a primitive root. If
    \(x \in F_{p}^{\times}\), then the \textbf{dicrete logarithm} of x base g
    is \(L \in \mathbb{Z}\) s.t.\ \(0 \leq L < p - 1\) and \(x \equiv g^{L}
    \pmod{p}\). The function is denoted as \(L_{g}(x)\). \\
    Some properties
    \begin{itemize}
      \item \(L_{g}(1) = 0\)
      \item \(L_{g}(g) = 1\)
      \item \(L_{g}(xy) \equiv L_{g}(x) + L_{g}(y) \pmod{p - 1}\)
      \item \(L_{g}(x^{-1}) \equiv -L_{g}(x) \pmod{p - 1}\)
      \item \(L_{g}(g^{a} \pmod{p}) = a \pmod{p - 1}\)
      \item \(g^{L_{g}(x)} = x \pmod{p}\)
    \end{itemize}

    \textbf{Problem}
    \textit{Discrete logarithm problem is given a (large) prime p, a primitive
    root g mod p, and an element \(x \in F_{p}\) calculate \(L_{g}(x)\)}.

  \end{subsection}

  \begin{subsection}{Diffie-Hellman}

    \(k_{E} = (p, g)\) where p is prime and g is a primitive root mod p. The
    prime needs to be large. The private key is shared between to communicating
    users, with \(k_{d} = g^{\alpha\beta} \pmod{p}\) with \(\alpha\) chosen by
    user 1 and \(\beta\) by user 2. It is important that user 1 does not know
    \(\beta\) only \(g^{\beta}\) and computes that knowing \(\alpha\)
    (identical situation for user 2). To find the private key, both users send
    \(g^{\alpha}\) and \(g^{\beta}\) over a public channel.

    As the prime p and primitive root g are public, they can be taken from a
    public database rather than be randomly generated like in RSA (beneficial
    for performace reasons).

    The private key is then used for some symmetric cryptographic encoding. In
    the real world, AES is usually used.

    The Diffie-Hellman problem (as well as the discrete logarithm problem) can
    also be introduced in any abelian group with \(k_{d} = [\alpha \beta] g\)
    and so on (for example elliptic curves, as seen later).

    \begin{subsubsection}{Security}

      To compute the private key, the attacker would need to solve the discrete
      logarithm problem mod \(p - 1\).

      The securinty of Diffie-Hellman is also dependent on the security of the
      symmetric encryption system used.

      \textbf{Baby step, giant step}
      algorithm can be used to find the discrete logarithm. The aim is to find
      L s.t.\ \(g^{L} \equiv x \pmod{p}\).
      First take \(N = \lceil \sqrt{p - 1} \rceil\) then make two lists
      \begin{itemize}
        \item \textbf{Baby step}: \(g^{j} \pmod{p}\) for \(j \in [0, N]\)
        \item \textbf{Giant step}: \(xg^{-Nk} \pmod{p}\) for \(k \in [0, N]\)
      \end{itemize}
      Then a match in the two lists has to be found. It is only neccessary to
      compute the entirety of the first list, then compute the second one until
      the match is found. Once there is a match there is \(g^{j} \equiv
      xg^{-Nk} \pmod{p}\) therefor \(g^{j + Nk} \equiv x \pmod{p}\) and
      \(L = j + Nk\).

      \textbf{Index Calculus}
      algorithm is another way to solve the discrete logarithm problem. Firstly
      find multiple (at least B)  B-smooth number (specify B) which are
      poewrs of g mod p, \(g^{i} * a \equiv b \pmod{p}\) with \(i \in
      \mathbb{N}\) and a, b smooth, \(a = 2^{a_{1}} 3^{a_{2}} \dots\) and
      \(b = 2^{b_{1}} 3^{b_{2}} \dots\). Then taking the discrete logarithm,
      \(i + a_{1} L_{g}(2) + a_{2} L_{g}(3) \dots \equiv b_{1} L_{g}(2) +
      b_{2} L_{g}(3) \pmod{p - 1}\). This can be used to determine \(L_{g}(2),
      L_{g}(3) \dots\). Then given x, calculate \(xg^{i}\) until it is smooth,
      then taking discrete logarithm \(L_{g}(x) + i \equiv c_{1} L_{g}(2) +
      c_{2} L_{g}(3) \dots \pmod{p - 1}\) where both \(c_{i}s\) and
      \(L_{g}(q)\) are known.

    \end{subsubsection}

  \end{subsection}

  \begin{subsection}{Elgamal}

    This system relies on the same math as Diffie-Hellman. The receiver chooses
    a prime p, g primitive root mod p and \(k_{D} = \alpha\). Then \(k_{E} =
    g^{\alpha}\). The sender chooses \(k \pmod{p - 1}\) and for a message m
    sends \((x, y) = (g^{k}, mk_{E}^{k})\). The receiver decodes message as
    \[m = yx^{-\alpha}.\] The random k implies that each message has many
    different decriptions. This means that tehre is no way to verify wherether
    m is correct if it is guessed. \\
    The security of this system is dependent on the same problem as
    Diffie-Hellman. \\
    A good choice of prime would be p s.t.\ \(p = 2q + 1\) where q is also
    prime.

    \begin{subsubsection}{Signature}

      For a public known \(p, g\) with p prime and g primitive root mod p, then
      for \(\alpha \in (\mathbb{Z} / p){}^{x}\) then the person signing sends
      \(y = g^{\alpha} \pmod{p}\). A valid signature for a message m a valid
      signature is a pair \((r, s)\) s.t.\ \[y^{r}r^{s} = g^{m} \pmod{m}.\]
      Each message has about p valid signatures out of about \(p^{2}\) possible
      pairs. This means signatures are not unique. Finding a valid signature
      for a message is belived to be a hard problem without knowledge of
      \(\alpha\). This signature can be broken with solutions to the dicrete
      logarithm. The signature is chosen by taking \[r \equiv g^{k} \pmod{p}\]
      \[s \equiv k^{-1}(m - \alpha r) \pmod{p - 1}\] for a random
      \(k \pmod{p - 2}\) and coprime to \(p - 1\). \\
      \textit{It is important that the sender does not use the same k twice}.

    \end{subsubsection}

  \end{subsection}

\end{section}

\begin{section}{Elliptic Curves}

  \begin{subsection}{Introduction}

    Take the curve \[y^{2} = x^{3} - 2.\] It only has two solution in integers,
    namely \((3, \pm 5)\). But is has infinetly many solutions in the
    rationals, which can be generated by
    \[\left( \frac{x^{4} + 16x}{4y^{2}}, \frac{-x^{6} + 40x^{3} + 32}{8y^{3}}
      \right)\]
    where \((x, y)\) is already a valid solution. Any line joining \(P_{1},
    P_{2}\) on the curve will also intersect it at a third point, unless it
    is a tangent or it is vertical. This allows for a definision of an
    operation \(P_{1} \# P_{2} = P_{3}\) with \(P_{3}\) being the third point
    intersected by the line joining the other two. Its reflection in the
    x-axis, denoted as \(P_{1} \oplus P_{2}\) forms an \textbf{abelian group}
    with all the rational points on this curve. \textit{Identity is not exacly
      on the curve, it is infinitely far along the y-axis, aslo have to take
      care of the outliers later}.

    An \textbf{elliptic curve} is an equation of the form
    \[E(F): y^{2} = x^{3} + ax + b\] with \(a, b \in F\) a field and
    \(\triangle_{E} = 4a^{3} + 27b^{2} \neq 0\). The non zero condition is to
    ensure \textit{smoothness}, which will not be defined properly. For a
    curve E over F, a point on E is either \((x, y) \in F^{2}\)
    (\textbf{normal point}) or \textbf{point at infinity}, writen as
    \(\mathcal{O}\). This point is inifinitely far along the y-axis in either
    direction. \\
    \textit{Generally E will denote \(E(\mathbb{Q})\)}.

  \end{subsection}

  \begin{subsection}{Algebraic Definision}

    For E an elliptic curve over F and \(P = (x_{p}, y_{p}),
    Q = (x_{q}, y_{q})\) point on E:\ (definision for \(\oplus\))
    \begin{itemize}
      \item \(P \oplus \mathcal{O} = \mathcal{O} \oplus P = P\)
      \item If \(x_{p} = x_{q}\) and \(y_{p} = -y_{q}\) then
            \(P \oplus Q = \mathcal{O}\)
      \item If \(x_{p} \neq x_{q}\) \\
            \(\lambda = \frac{y_{q} - y_{p}}{x_{q} - x_{p}}\) and
            \(\mu = y_{p} - \lambda x_{p} = y_{q} - \lambda x_{q}\) \\
            \(x_{r} = \lambda^{2} - x_{p} - x_{q}\) \\
            \(y_{r} = \lambda x_{r} + \mu\) \\
            \(P \oplus Q = R = (x_{r}, y_{r})\)
      \item If \(x_{p} = x_{q}\) and \(y_{p} = y_{q} \neq 0\) \\
            \(\lambda = \frac{3x^{2}_{p} + a}{2y_{p}}\) and
            \(\mu = y_{p} - \lambda x_{p} = y_{q} - \lambda x_{q}\) \\
            \(x_{r} = \lambda^{2} - x_{p} - x_{q} = \lambda^{2} - 2x_{p}\) \\
            \(y_{r} = \lambda x_{r} + \mu\) \\
            \(P \oplus Q = (x_{r}, -y_{r})\)
    \end{itemize}

    The F-set of points on E with \(\oplus\) forms an \textbf{abelian group} as
    said above, with inverses \(P^{-1} = -P = (x_{p}, -y_{p})\). Powers are
    dentoted as \([n]p = p \oplus p \oplus \cdot \oplus p\) n times.

  \end{subsection}

  \begin{subsection}{Torsion (Change title later maybe?)}

    For G an \textbf{abelian group}, if an element \(g \in G\) is
    \textbf{cyclic}, \([n]g = e\) for \(n \in \mathbb{N}\) it is called
    \textbf{torsion} or more specifically \textbf{n-torsion} if n is specified.
    \(G_{tors}\) denotes the entire \textbf{subgroup} of torsions
    \(g_{tors} = \{g \in G | [n]g = e, \; n \in \mathbb{N}\}\). \(G[n]\)
    denotes a similar \textbf{subgroup} with only n-torsions.

    \textit{In a finite group, all elements are torsion.}

    \textit{Non examinable but fun} \\
    \(E(\mathbb{C})\) is \textbf{isomorphic} to \(\mathbb{C} / Q\) where
    \(Q = {x + iy | x, y \in \mathbb{Z}}\). This makes this elliptic curve a
    topological torus (coffe mug). More importantly
    \[E(\mathbb{C})[n] \cong (\mathbb{C} \ Q)[n] = \{z \in \mathbb{C}
      | nz \in G\} \cong (\mathbb{Z} / n){}^{2}.\]

    For \(P = (x, y) \in E(\mathbb{Q})\), then if P is a \textbf{torsion}
    \((x, y) \in \mathbb{Z}^{2}\) and \(y = 0\) (2-torsion) or \(y^{2} |
    \triangle\). The converse of this statement is not neccessarly true.
    \(E(\mathbb{Q}){}_{tors}\) is finite. In fact \(E(\mathbb{Q}{}_{tors})\) is
    either
    \begin{itemize}
      \item Cyclic of order \(N \in [1, 10]\) or 12
      \item Of a form \(\mathbb{Z} / 2 \times \mathbb{Z} / 2N\) for \(N \in
            [1, 4]\)
    \end{itemize}
    All of the above cases exist.

    For G \textbf{finitely generated} abelian, \(G \cong G_{tors} \times
    \mathbb{Z}^{r}\) where r is called the \textbf{rank} of G, \(r \neq 0
    \Rightarrow\) G not finite. \(E(\mathbb{Q})\) is
    \textbf{finitely generated}.


  \end{subsection}

  \begin{subsection}{In Finite Fields}

    For this we are interested in \(E(\mathbb{F}_{p})\) with \(p > 2\)
    (\(p = 2^{n}\) is often used in practice). This group can be used for
    \textbf{Diffie Hellman} problem similarly to how the group
    \((\mathbb{Z} / p, \times)\) was used in the section above. Such group is
    finite with \(| p + 1 - \# E(\mathbb{F}_{p}) | \leq 2 \sqrt{p}\). They are
    isomorphic to either
    \begin{itemize}
      \item A cyclic group \(\mathbb{Z} / m\)
      \item \(\mathbb{Z} / m \times \mathbb{Z} / n\) with m, n cooprime
    \end{itemize}

    If the \(E(\mathbb{F}_{p})\) has prime order, it is finitely generated by
    any non zero element. If it is isomorphic to \(\mathbb{Z} / m\), the
    generator is any g s.t.\ \([d]g \neq \mathcal{O}\) for \(d | m, d < m\).
    Otherwise it can be generated by any \(g_{1}, g_{2}\) s.t.\ \([d]g_{1} +
    g_{2} \neq \mathcal{O}\) for any d.

    \begin{subsubsection}{Diffie Hellman}

      As mentioned above, elliptic curve cryptography also uses Diffie-Hellman
      key exchange in a different group than originally described. The
      information exchanged is an elliptic curve \(E(\mathbb{F}_{p})\) and a
      point \(P \in E\) s.t.\ P has a large prime order n. Both users have some
      private information \(\alpha, \beta \in [0, n - 1]\), then send their
      public keys \(k_{E} = [\alpha] P\) for user 1 and \(k_{E} = [\beta] P\)
      for user 2. This results in the private key \(k_{D} = [\alpha \beta] P\).

      It is belived that the discrete logarithm problem, on which the security
      of this relies on, is even harder in an elliptic curve group than in
      \(\mathbb{Z} / p\)

    \end{subsubsection}

    \begin{subsubsection}{Elgamal Signatures}

      Similarly to Diffie Hellman, both ends agree on E elliptic curve and P a
      generaotr. A valid signature for a message \(m \in \mathbb{Z} / n\) is a
      pair \(R, s\) s.t.\ \[[R_{x}] Q \oplus [s] R  = [m] P\] with
      \(Q = [\alpha] P \), the public key of the sender and \(R_{x}\) denoting
      the x coordinate of R. Similarly to basic Elgamal, to generate the pair
      the sender first need a random integer \(k \in \mathbb{Z} / n\) (new one
      for each signature) then
      \[R = [k] P \; and \; s = k^{-1}(m - R_{x} * \alpha).\]

    \end{subsubsection}

  \end{subsection}

  \begin{subsection}{Schoof's Algorithm}

    \textbf{Not from lecture notes, not examinable} \\
    Main source : https://www-users.cse.umn.edu/~musiker/schoof.pdf

    The basic aim is to compute the number of points of a given curve
    \(E(\mathbb{F}_{q})\). This is achived by computing \(| E(\mathbb{F}_{q}) |
    \pmod{l}\) for all prime \(l \in S\) s.t.\ \(\Pi_{l \in S}l = N > 4
    \sqrt{q}\). This will allow computing \(| E(\mathbb{F}_{q}) | \pmod{N}\) by
    the Chinese Remaindr Theorem which will in turn allow deducing the size of
    the elliptic curve due to its  restircted forms. For the rest of this
    subsection let \[t_{l} = q + 1  - | E(\mathbb{F}_{q}) | \pmod{l}.\]

    \newpage

    \begin{subsubsection}{The Characteristic Equation for the Forbenius
        Endomorphism}

      For an elliptic curve \(E(\mathbb{F}_{q})\) let \(\overline{E}\) define the
      same elliptic curve over \(\overline{\mathbb{F}_{q}}\), the algebraic closure
      of \(\mathbb{F}_{q}\). There exists \(\pi : \overline{E} \to \overline{E}\) for
      which \(\pi(x, y) = (x^{q}, y^{q})\). \(\pi\) also fixes E and satisfies
      the characteristic equation \(\pi - 1 = 0\). Meanwhile on all of
      \(\overline{E}\) it satisfies the quodratic \(\pi^{2} - t \pi + q = 0\) where
      \(t = q + 1 - | E(\mathbb{F}_{q}) |\). \\
      Plugging \(P = (x, y) \in \overline{E}\), we get \[(x^{q^{2}}, y^{q^{2}})
        \oplus q(x, y) = t(x^{q}, y^{q}).\] If \(P \in \overline{E}[l]\) then for any
      \(r \in \mathbb{Z}\) \(rP = \overline{r}P\) where \(\overline{r} \equiv r
      \pmod{l}\). Thus for \((x, y) \in \overline{E}[l]\) the equation can be
      reduced to
      \[(x^{q^{2}}, y^{q^{2}}) \oplus \overline{q}(x, y) \equiv
        \overline{t}(x^{q}, y^{q}) \pmod{l}.\]

    \end{subsubsection}

    \begin{subsubsection}{Division Polynomials}

      Polynomials \(\psi_{0}, \psi_{1}, \dots\) are defined as a recurance
      relation with the polynomials 0 to 4 predefined. They have the remarkable
      property that every torsion point \(p \in \overline{E}[n] / \{ \mathcal{O} \}\)
      satisfies \(\psi_{n}(x_{p}, y_{p}) = 0\). In fact for \(E(F) : y^{2} =
      x^{3} + Ax + B\) with \(char(F) \neq 0\) and \((x, y) = P \in E\)
      \[[n]P = \left( \frac{\phi_{n}(x)}{\psi_{n}^{2}(x)},
        -\frac{\omega_{n}(x, y)}{\psi_{n}^{3}(x, n)} \right)\]
      where \[\phi_{m} = x \psi_{m}^{2} - \psi_{m + 1} \psi_{m - 2}\]
      \[\omega_{m} = \frac{\psi_{m + 2} \psi_{m - 1}^{2} - \psi_{m - 2}
          \psi_{m + 1}^{2}}{4y}.\]
      \textit{Also for \([n]P = (a, b)\), \([-n]P = (a, -b)\), which is quite clear
        from inverses on elliptic curves.}

      \(\psi_{n}^{2}, \frac{\psi_{2n}}{y}, \psi_{2n + 1} \; and \; \phi_{n}\)
      are a function in x alone and \(\psi^{2}\) has degree \(n^{2} - 1\),
      which is equal to the number of finite n-torsion points.

      \((x_{\overline{q}}, y_{\overline{q}}) = [\overline{q}](x, y) = \)

    \end{subsubsection}

    \begin{subsubsection}{Title needed}

      For \((x', y') = (x^{q^{2}}, y^{q^{2}}) \oplus \overline{q}(x, y)\) where
      \(p_{1} \neq \pm p_{2}\) is assumed as the opposite is very rare,
      \[x' = \left ( \frac{y^{q^{2}} - y_{\overline{q}}}{x^{q^{2}} -
            x_{\overline{q}}} \right ){}^{2} - x^{q^{2}} - x_{\overline{q}}.\]

    \end{subsubsection}

  \end{subsection}

\end{section}

\newpage

\begin{section}{Codes Basics}

  \begin{subsection}{Basic Concepts}

    An \textbf{alphabet} is a finite set of symbols. For an alphabet A,
    \(A^{n}\) denotes the set of all lists of n symbols from A (this implies
    \(| A |^{n} = | A^{n} |\)). Each element of this list is called a
    \textbf{word} of length n. A \textbf{code} C of \textbf{block length} n on
    the alphabet A is a subset of \(A^{n}\) (\(C \subset A^{n}\)). A
    \textbf{codeword} c is an element \(c \in C\). \\
    If \(| A | = 2\), then C is called a \textbf{binary} code.

    \begin{subsubsection}{Basic Model}

      The sender sends a message which is encoded into a codeword \(c \in C\).
      This codeword is send over a \textbf{channel} which may or may not
      distort the codeword (change some symbols). The codeword is then received
      by the end user and \textbf{decoded} to (hopefully) the original message.
      The decoder executes two actions (1) guessing what codeword was intended
      to be sent, (2) decode the codeword into the actuall message.

    \end{subsubsection}

    \begin{subsubsection}{Hamming Disctance}

      For some alphabet A, and \(x, y \in A^{n}\), the \textbf{Hamming}
      distance between x and y (written \(d(x, y)\)), is the number of
      places in which x differes from y. In context of codes, if x is sent and
      y is received, \(d(x, y)\) denotes the \textbf{symbol-errors} that
      have occured. As Hamming distance is a proper distance measure, it
      satisfies the following conditions
      \begin{itemize}
        \item \(0 \leq d(x, y) \leq n\)
        \item \(d(x, y) = 0 \iff x = y\)
        \item \(d(x, y) = d(y, x)\)
        \item \(d(x, y) \leq d(x, z) + d(y, z)\)
      \end{itemize}

    \end{subsubsection}

    \begin{subsubsection}{Code Properties}

      For a code C, its minimum distance, \(d(C)\), is defined as the smallest
      distance between two distinct codewords in C.

      For a code C with block length n, M codewords and minimum distance of d
      is called an \((n, M, d)\) code or for (\( | A | = q\)) the code over A
      is called \((n, M, d){}_{q}\) code.

    \end{subsubsection}

  \end{subsection}

  \begin{subsection}{Nearest Neighbour Decoding}

    \begin{subsubsection}{Definision}

      For a code \(C \subset A^{n}\) and \(x \in A^{n}\), the \textbf{nearest
        neighbour} of x is a codeword \(c \in C\) s.t.\ \(d(x, c) = \min \{
      d(x, y) | y \in C \}\). A word may have multiple nearest neighbours.

    \end{subsubsection}

    \begin{subsubsection}{Decoding}

      \textbf{Nearest neighbour decoding} is a decoding scheme which maps the
      recieved word x into its nearest neighbour in C. This can always be done
      by computing \(d(x, c) \; \forall c \in C\), but latter better methods
      will be shown. If x has multiple nearest neighbours, it is interpretted
      as an error and appropriate action is taken (such as request of
      retransmission).

      For this decoding, if a message x is received with t erros,
      \begin{itemize}
        \item If \(t \leq d - 1\) then it can be detected an error has occured
        \item If \(t \leq \lfloor \frac{d - 1}{2} \rfloor\) then the decoding
              will correct them
      \end{itemize}
      More errors may be detected/corrected but this is the guarenteed minimum
      performance.

    \end{subsubsection}

  \end{subsection}

  \begin{subsection}{Probability}

    The entire course assumes that a codeword is more likely to suffer a
    smaller amount of errors. Let p represent the probability for a
    \textbf{symmetric} channel s.t.\
    \begin{itemize}
      \item For any \(a \in A\) the probability that it is changed is p
      \item For any \(a, b \in A\), \(a \neq b\), the probability that a is
            changed to b is \(\frac{p}{q - 1}\)
    \end{itemize}

    Symmetric channels are easy to work with but many real world scenarios are
    not strictly symmetric. This would affect the second statment.

    Then for a codeword \(c \in C\) is a q-array code of block length n, sent
    over a symmetric channel with symbol-error probability p, \\ \(P(x \;
    received |  c \; sent) = (\frac{p}{q - 1}){}^{t} (1 - p){}^{n - t}\) where
    \(t = d(c, x)\). This implies that the probability for no errors is
    \((1 - p){}^{n}\). Also if \(p < \frac{q - 1}{q}\), then messages have
    smaller chance to have more errors. Any channel where \(p =
    \frac{q - 1}{q}\) has an equal probability of receiving any symbol no
    matter what is send, therefor being a random channel that is not worth
    considering. The other important probability, \(P(c \; sent
    | x \; received)\) also increases as \(d(x, c)\) decrease, under
    assumption that all messages have an equal chance of being sent.

  \end{subsection}

  \begin{subsection}{Aims}

    What makes a good \((n, M, d){}_{q}\) code?
    \begin{itemize}
      \item Smaller n will allow messeges to be transmitted faster
      \item Larger M will provide more words allowing more info to be
            communmicated
      \item Larger d will allow to correct and/or detect more symbol errors
    \end{itemize}
    But all of these parameters are related to each other. The
    \textbf{Singleton Bound} states that \(M \leq q^{n - d + 1}\). A code
    which satisfies this bound strictly is know as \textbf{Maximum Distance
      Seperable} or MDS.

    For an alphabet A s.t.\ \(| A | = q\), \(1 \leq t \leq n\) and
    \(x \in A^{n}\)
    \begin{itemize}
      \item \(S(x, t) = \{ y \in A^{n} | d(x, y) \leq t\}\) is a sphere of
            radious t around x
      \item A code \(C \subset A^{n}\) is called \textbf{perfect} if
            \(\exists t\) s.t.\ \(A^{n}\) is the disjoint union of all the
            \(S(c, t) \;\; \forall c \in C\)
    \end{itemize}

    To figure out whenether a code is \textbf{perfect} it is not actually
    neccessary to consider the codewords. For any \(x \in A^{n}\)
    \[| S(x, t) | = \sum^{t}_{k = 0} {{n}\choose{k}} (q - 1){}^{k}.\]
    The size of S can then be compared to the size of the alphabet, and the
    code and be perfect if \(| S | | | A^{n} |\).

    \newpage

    Another bound to consider is the \textbf{Hamming Bound} (or the
    \textbf{Sphere-packing Bound}). Namely
    \[M \sum^{t}_{k = 0} {{n}\choose{k}} (q - 1){}^{k} \leq q^{n}\] where
    \(t = \lfloor \frac{d - 1}{2} \rceil\). \\
    This can be alternitevly writen as
    \[M | S(x, \lfloor \frac{d - 1}{2} \rceil) | \leq q^{n}.\]
    The \textbf{Hamming Bound} holds equal iff the code is \textbf{perfect}.
    A perfect code must have d odd.

  \end{subsection}

\end{section}

\begin{Linear Codes}

  \begin{subsection}{Codes as Subspaces}

    From now on, assume A is a finite field. Then a word is a vector \(w \in
    A^{n}\). The code C has to be a \textbf{subspace} of \(A^{n}\). A size of a
    vector space \(| \F^{n}_{q} | = q^{n}\). \\
    A \textbf{linear code} is subspace of \(\F^{n}_{q}\) for some finite field
    \(\F_{q}\) and non negative integer n. \(\textbf{0}\) is a part of every
    subspace.

    \begin{subsubsection}{Spanning}

      A code C is said to \textbf{span} \(\langle C \rangle\). If it is closed
      then \(\langle C \rangle = C\) and it can be a linear code.

    \end{subsubsection}

    \begin{subsubsection}{Weight}

      For a vector \textbf{x}, its weight, \(w(\textbf(x))\) is definded as
      the number of non-zero elements in \textbf{x}. Also for \textbf{hamming}
      distance, \(d(\textbf{x}, \textbf{y}) = w(\textbf{x} - \textbf{y})\).

\end{document}
