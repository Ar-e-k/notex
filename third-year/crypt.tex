\documentclass[12pt, letterpaper]{article}

\usepackage{graphicx}
\usepackage{parskip} % Disabling paragraph index as it does not fit maths
\usepackage{hyperref} % Usable menu and references

\graphicspath{{images}}

\title{Cryptography}
\author{Arkadiusz Naks}
\date{2023}

\begin{document}

\begin{titlepage}
  \begin{center}
    \makeatletter
    \vspace*{1cm}
    \Huge
    \textbf{\@title}

    \vspace{0.5cm}
    \Large
    Lecture notes from Cryptography half of Cryptography and Codes module at Durham University

    \vspace{1.5cm}

    \textbf{\@author}

    % \includegraphics[scale=0.55]{.png}
    \vfill

    \vspace{0.8cm}

    \small
    Based on my understanding of lectures and notes of \\
    \@date{}
  \end{center}
\end{titlepage}

\tableofcontents
\newpage

\begin{section}{Important Definisions}

  A place for short and important definisions \\

  \textsc{Definision} ()

\end{section}

\begin{section}{Aims}

  There are multiple reasons of varying importance for encrypting comunication
  \begin{itemize}
    \item \textbf{Secrecy}: Hide content from outside viewers
    \item \textbf{Integrity}: Ensure the message cannot be tampered with
          (without the receiving side knowing)
    \item \textbf{Authenticity}: Similar to \textbf{integrity}, ensure that the
          receiver knows the message comes from where it claims it does
    \item \textbf{Non-Repudiation}: Make sure the sender cannot claim they did
          notes send the message
  \end{itemize}
  \textbf{Kerchhoffs's principle}: It is important cryptographic systems are
  still secure iven if the enemy knows the details of their method.
  There are multiple methods of attack, with the main catigories being:
  \begin{itemize}
    \item \textbf{Ciphertext-only}: Enemy obtains ciphertext and deduces the
          plaintext. Mostly relevant in naive systems
    \item \textbf{Know plaintext}: Enemy knows some plaintext and ciphertext and
          uses it to deduce the key
    \item \textbf{Chosen plaintext}: Enemy tricks the sender to encode some
          chosen planetext and observes the resultant cyphertext (similar to
          \textbf{Known plaintext})
    \item \textbf{Chosen ciphertext}: Enemy trickes the receiver into decoding
          some chosen ciphertexts and observers the result (\textbf{Chosent
          plaintext})
    \end{itemize}

\end{section}

\end{document}
