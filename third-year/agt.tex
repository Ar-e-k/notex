\documentclass[12pt, letterpaper]{article}

\usepackage{graphicx}
\usepackage{parskip} % Disabling paragraph index as it does not fit maths
\usepackage{hyperref} % Usable menu and references
\usepackage{amssymb} % Used to show sets of sumbers, like the real numbers

\begin{document}

\begin{section}{Introduction}

  A game models a situation where tow or more players have to take some
  decisions that will influence their welfare. The \textbf{payoff} of each
  player can depend on decisions of all players not just their own. \\
  A game describes a strategic interaction that inclueds constraints on the
  actions the players can take as well as their interests, but not the specific
  actions. A \textbf{solution} is a systematic description of the outcomes that
  may emerge in a family of games.

  There are four basic groups of games
  \begin{itemize}
    \item Strategic games
    \item Extensive games with perfect information
    \item Extensive games without perfect information
    \item Coalition games
  \end{itemize}
  In the first three of these, the sets of possible actions of individual
  players are \textbf{primitive} (non cooperative). In coalition games, the
  sets of possible joint actions of groups of players are \textbf{primitives}.

  A \textbf{strategic} game model, each player chooses the plan of actoins once
  and for all, with all decisions being made simultaneously. When choosing the
  plan, players are not informed of others decisions. \\
  On the other hand in \textbf{extensive} game models, each player can consider
  their actoins at each decision.

  Perfect and imperfect information games are very much self expanotrary
  (however that is spelled).

  In game theory, players are assumed to be
  \begin{itemize}
    \item \textbf{Rational}: decisions are consistent to achive well defined
          objectives
    \item \textbf{Intelligent}: all informed about how the game operates,
          can make any inferences, takes into account the same rulles for other
          players
  \end{itemize}

  \begin{subsection}{Strategic Games}

    \begin{subsubsection}{Formal Definision}

      A \textbf{strategic game} is defined by
      \begin{itemize}
        \item A set of players N
        \item A set of sets of actons \(\{S_{i} | \forall i \in N\}\), S
        \item A set of \textbf{payoff functions} \(\{u_{i} | \forall i \in
              N\}\) where each \(u_{i}: S_{i} \to \mathbb{R}\), U
      \end{itemize}

      This is usually denoted as \(\lceil = \langle N,
      (S_{i}){}_{\forall i \in N}, (u_{i}){}_{\forall i \in N} \rangle\).

    \end{subsubsection}

    \begin{subsubsection}{Symetric Games}

      A strategic game is \textbf{symetric} if
      \begin{itemize}
        \item Each player has the same set of actions; \\
              for any \(i, j \in N\), \(s_{i}, s_{j} \in S\) \(s_{i} = s_{j}\)
        \item All players have the same outcomome function; \\
              for any \(i, j \in N\), \(u_{i}, u_{j} \in U\) \(u_{i} = u_{j}\)
      \end{itemize}

    \end{subsubsection}

    \begin{subsubsection}{Strategies}

      A \textbf{mixed strategy} \(p_{i}\) for a player \(i \in N\) is a
      probability disctribution over the set of actions; \(p_{i} : S_{i} \to
      [0, 1]\) s.t.\ \(\sum_{s_{i} \in S_{i}} p_{i}(s_{i}) = 1\). The set of
      strategies for a player is denoted by \(\triangle (S_{i})\). \\
      A \textbf{pure} strategy is a strategy s.t.\ for one \(s_{i} \in S_{i}\);
      \(p_{i}(s_{i}) = 1\) (the player is certain to pick one action).

    \end{subsubsection}

    \begin{subsubsection}{Profiles}

      An \textbf{action profile} \(\textbf{s} \in S\) is a touple containing a
      single action (\textbf{pure strategy}) out of each set \(S_{i}\) (one
      action per player).

      A \textbf{strategy profile} is a touple containing one \textbf{mixed
        strategy} per player, denoted \(\textbf{p}_{i} \in \triangle(S_{i})\).
      \\
      Let \(\times\) denoted all possible \textbf{strategy profiles}
      \(\{ \triangle(S_{i}) | \forall i \in N \}\).

      \(\textbf{p}_{-i}\) (same for s) will denote a stategy profile with
      player i remomved and \((\textbf{p}_{-i}, p_{i})\) will denote a touple
      with the stategy for player i replaced with a different strategy.

    \end{subsubsection}

    \begin{subsubsection}{Expected Payoffs}

      Given a \textbf{strategy profile} \(\textbf{p} = p_{i \forall i \in N}\),
      the \textbf{expected payoff} of a player \(i \in N\) is the expected
      value of their payoff funcion. This can be calculated by
      \[u_{i}(\textbf{p}) =
        \sum_{s_{1}}( \cdots (\sum_{s_{n}}(
        \prod^{n}_{j=1} p_{j}(s_{j}) u_{i}((s_{1}, \dots , s_{n}))))).\]
      This essentially calculates the probability of each event times the
      payoff of that event for the given player. \\
      Alternative slide says \[u_{i}(\textbf{p}) = \sum_{s_{i} \in S_{i}}
        p_{i}(s_{i}) u_{i}((\textbf{p}_{-1}), s_{i}).\]

    \end{subsubsection}

    \begin{subsubsection}{Best Response}

      For a player i, their best response is, given a partial strategy
      \(\textbf{p}_{-i}\), is a strategy s.t.\ \(u_{i}((\textbf{p}_{-i}, s))\)
      is maximised.

    \end{subsubsection}

    \begin{subsubsection}{Support}

      The \textbf{support} of a stategy \(p_{i}\) of a player i is the subset
      of actoins of i where \(p_{i}\) poses strictly positive probability.
      \(Support(p_{i}) = \{ s_{i} \in S_{i} | p_{i}(s_{i}) > 0\}\).

    \end{subsubsection}

    \begin{subsubsection}{Domination}

      A strategy \(p_{i}\) \textbf{Strict Dominates} an action \(s_{i}\) if it
      is better than that action no matter what the other players choose.
      Formaly \(u_{i}((\textbf{s}_{-i}, p_{i})) >
      u_{i}((\textbf{s}_{-i}, s_{i}))\) for all \(\textbf{s} \in \times\).
      Any Nash equilibria has 0 probability for any \textbf{strictly dominated}
      action.

      A \textbf{Weakly dominated} strategy is the same except for the
      possibility equality.

      \textbf{Dominated} strategies do not need to exist. \\ Every finite game
      Nash equilibria in which no strategy is \textbf{Weakly} dominated.

    \end{subsubsection}

  \end{subsection}

  \begin{subsection}{Nash Equilibrium}

    In a Nash equilibrium each player chooses the best avaliable action given
    the choices of other players. This makes a Nash equilibrium is a
    combination of actions for all player, s.t.\ no player can increase their
    payoff by changing the action.

    \begin{subsubsection}{Pure Equilibrium}

      This is an action profile \textbf{s} s.t.\ it is a Nash equilibrium; any
      player i cannot choose a different action \(s_{i}' \notin \textbf{s}\)
      s.t.\ \(u_{i}(s_{i} \in \textbf{s}, \textbf{s} / \{s_{i}\}) <
      u_{i}(s_{i}', \textbf{s} / \{s_{i}\})\)

      Not every strategic game has a pure Nash equilibrium. A game can also
      have multiple Nash equilibria.

    \end{subsubsection}

    \begin{subsubsection}{Mixed Equilibrium}

      This is a generalisation of pure Nash equilibrium s.t.\ each player
      chooses a \textbf{mixed strategy} rather than a \textbf{pure strategy}.
      Formaly a \textbf{mixed strategy} is a strategy profile \textbf{p} s.t.\
      no single player can change their strategy profile which would result in
      a higher payoff.

      In a \textbf{symetric} game, a if for each player i, the strategy is the
      same, the resultant Nash equilibrium is called \textbf{symetric}.

      Every strategic game with finite number of players and finite action sets
      possesses at least one mixed Nash equilibrium. Although this is true, the
      problem of computing them for some games is PPAD-complete (hard??), even
      for some involving only two players. \\
      A symetric game always has at least one symetric equilibrium. \\
      Many games with infinte also possess Nash equilibria.

      A stategy \textbf{p} is a Nash equilibria iff for all players i and all
      \(s_{i} \in S_{i}\), \(s_{i} \in Support(p_{i}) \Rightarrow s_{i} \in
      \arg \max_{s_{j} \in S_{i}}((\textbf{p}_{-i}, s_{j}))\). This is
      equivelant to saying that it is only a Nash equilibria if for each player
      all their actions with non zero probability result in the same outcome.
      \\
      Another equivelant notion is that in a Nash equilibrium, each player
      plays the best response.

    \end{subsubsection}

  \end{subsection}

\end{section}

\end{document}
