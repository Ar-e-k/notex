\documentclass[12pt, letterpaper]{article}

\usepackage{graphicx}
\usepackage{parskip} % Disabling paragraph index as it does not fit maths
\usepackage{hyperref} % Usable menu and references
\usepackage{amssymb} % Used to show sets of sumbers, like the real numbers
\usepackage{amsmath} % Used for column vectors
\usepackage{xargs} % Used for multiple deafult command values

\graphicspath{{images}}

\newcommand{\R}{\mathbb{R}}
\newcommand{\C}{\mathbb{C}}
\newcommand{\Q}{\mathbb{Q}}
\newcommand{\Z}{\mathbb{Z}}
\newcommand{\F}{\mathbb{F}}
\newcommand{\N}{\mathbb{N}}
\newcommand{\Ok}{\mathcal{O}}
\newcommand{\p}{\mathfrak{p}}
\newcommand{\id}[1]{\mathfrak{#1}}

\newcommandx{\val}[1][1=\cdot]{| #1 |}
\newcommandx{\pdc}[3][1=a, 2=k, 3=0]{p^{#2} (\sum^{\infty}_{i = #3} #1_{i} p^{i})}

\title{Number Theory}
\author{Arkadiusz Naks}
\date{2023}

\begin{document}

\begin{titlepage}
  \begin{center}
    \makeatletter
    \vspace*{1cm}
    \Huge
    \textbf{\@title}

    \vspace{0.5cm}
    \Large
    Lecture notes from Number Theory at Durham University

    \vspace{1.5cm}

    \textbf{\@author}

    % \includegraphics[scale=0.55]{.png}
    \vfill

    \vspace{0.8cm}

    \small
    Based on my understanding of lectures and notes of \\
    \@date{}
  \end{center}
\end{titlepage}

\tableofcontents
\newpage

\begin{section}{Important Definisions}

  A place for short and important definisions \\

  \textsc{Definision} (Diophantine equation) \textit{Polynomial equations with
    integer coefficients}

  \textsc{Definision} (Regular Prime) \textit{A prime p which does not difide
    the order of the ideal class group of \(\Q(\xi_{p})\) with \(\xi_{p}\)
    being the pth root of unity}

\end{section}

\begin{section}{Introduction}

  Most classic example of a question solved through number theory is the famous
  Fermat's Last Theorem (FLT) (actually a conjecture), which states that
  \[x^{n} + y^{n} = z^{n}\] has no solutions in \(\Z\) excluding 0 when
  \(n > 0\). The proof is very complex and therefor omited. One aim of varphis
  course will be to solve \textbf{diophantine} equations with integer or
  rational solutions.

\end{section}

\begin{section}{Divisibility}

  \begin{subsection}{Euclidean Domains}

    Every ED is a \textbf{principal ideal domain}. \\
    An \textbf{euclidean domain} is a ring R which has a
    \textbf{euclidean function} \(\varphi\). This function is defined
    as \(\varphi: R \backslash{} \{0\} \to \N_{0}\) such that:
    \begin{itemize}
      \item \(\forall x, y \in R \backslash{} \{0\}\) \(\varphi(x) \leq \varphi(xy)\)
      \item \(\forall x \in R, y \in R \backslash{} \{0\} \exists q, r\) s.t.
            \(x = qy + r\) with \(\varphi(r) < \varphi(y)\)
    \end{itemize}

    Some examples of EDs are:
    \begin{itemize}
      \item \(\Z\) with \(\varphi: x \to |x|\)
      \item \(F[x]\) where F is a field with \(varphi: f(x) \to \deg(f)\)
      \item \(\Z[\sqrt{-2}]\), with \(\varphi: a + b\sqrt{2}
            \to a^{2} + 2b^{3}\)
    \end{itemize}

  \end{subsection}

\end{section}

\begin{section}{Field Extensions}

  \begin{subsection}{Basic Definision}

    For fields F and L, if F is contained in L and they have the same operations,
    F is called a \textbf{subfield} of L and L is a \textbf{field extention} of
    F, denoted as \(L/F\). Every \textbf{field extension} \(L/F\) implies L is a
    vectorspace over F, with usual vectorspace properties
    \begin{itemize}
      \item \(0 \in L\)
      \item \(u, v \in L \to u + v \in L\)
      \item \(a \in F\) and \(u \in L \to au \in L\)
    \end{itemize}
    (Unsure why this is useful as \(a \in F \to a \in L\) and a field is closed
    under both addition and multiplication)

    Each field extension \(L/F\) has a \textbf{degree} \([L : F]\) which s the
    dimension \(\dim_{F}L\) and can be infinite. If the dimension is finite, \(L/F\)
    is called a \textbf{finite field extension}.

  \end{subsection}

  \begin{subsection}{Algebraic Field Extensions}

    For a field extension \(L/F\), any element \(\alpha\in L\) is called
    \textbf{algebraic over F} if \(\exists f(x) \in F[x], f(x) \neq 0\) s.t.
    \(f(\alpha) = 0\). If all elements of L are algebraic over F, \(L/F\) is
    called an \textbf{algebraic extension} (alternitevly it is said that L is
    algebraic over F). \\
    \textit{Only a finite dimention field extension can be algebraic.} \\
    In fact all \textbf{finite} field extensions are \textbf{algebraic}.

    For \(\alpha \in L\) algebraic, its \textbf{minimal polynomial}
    \(p_{\alpha}(x)\) (or more precisely \(p_{\alpha, F}(x)\) when the field is
    not clear) is defined to be the monic polynomial of smallest degree s.t.\
    \(\alpha\) is the root. This always exists as \(\alpha\) is algebraic. It is
    also always unique. This polynomial is useful as its coefficients \(\in F\)
    containe information about \(\alpha \in L\). The polynomial is also always
    irreducable over F. \\
    Morever if \(f(\alpha) = 0\), f is monic and irreducable, it is the
    \textbf{minimal polynomial} of \(\alpha\). \\
    \textit{The \textbf{degree} of \(\alpha\) is defined as the \textbf{degree}
      of its minimal polynomial}.

  \end{subsection}

  \begin{subsection}{Generated Fields}

    For a \textbf{field extension} \(L / F\) and \(\alpha \in L\) we define
    \(F(\alpha) \subseteq L\) to be the smallest field extension of F which
    contains \(\alpha\); \(F(\alpha) = K\) where K is a field, \(\alpha \in K\)
    and \(F \subseteq K \subseteq L\) (K is the smalles subfield of L which
    contains both F and \(\alpha\)). Note \(\alpha \notin F\). \(F(\alpha)\)
    is called the \textbf{field generated by \(\alpha\) over F}. \\
    More generally \(F(\alpha_{1}, \dots , \alpha_{n}) = F(\alpha_{1})
    (\alpha_{2}) \dots F(\alpha_{n})\). \\
    It is important to remember that generally \(F(\alpha) \neq F[\alpha]\) as
    if \(\alpha\) is not \textbf{algebraic} (over F), \(\alpha^{-1} \notin
    F[\alpha]\) and \(F[\alpha]\) is not a field. On the other hand, if
    \(\alpha\) is \textbf{algebraic}, \(F[\alpha] = F(\alpha)\). In such case,
    the degree of \(F(\alpha) / F\) is the same as the degree of \(\alpha\),
    \([F(\alpha) : F]= \deg p_{\alpha, F}\). \\
    If \(K / F\) and \(L / K\) are field extensions, s.t. \(F \subseteq K
    \subseteq L\), then it is called a \textbf{tower} of fields and
    \([L : F] = [L : K] \times [K : F]\) (if one side is infinite, so is the
    other).

  \end{subsection}

  \begin{subsection}{Norm and Trace}

    Let \(L / F\) finite field extension and \(n = [L : F]\). Then as L is a
    vector space over F for any \(\alpha \in L\) there is a F-linear map
    \[\hat \alpha: L \to L, \; x \mapsto \alpha x.\]
    Then for a basis of L over F \((\beta_{i})_{i \in [1, n]}\) and a
    \(n \times n\) matrix \(T_{\alpha} = T_{\alpha, L / F} \in M_{n}(F)\) is
    defined to be \[\hat \alpha(\beta_{i}) = \alpha \beta_{i} =
      t_{i1} \beta_{i} + \cdots + t_{in} \beta_{i}\] where \(t_{ij} \in F\) and
    \(T_{\alpha} = (t_{ij})\). This means that
    \[\alpha \begin{pmatrix} \beta_{1} \\ \vdots \\ \beta_{i} \end{pmatrix} =
      T_{\alpha} \begin{pmatrix} \beta_{1} \\ \vdots \\ \beta_{i} \end{pmatrix}
      ,\]
    and therefore \(\alpha\) is an \textbf{eigenvalue} of the matrix
    \(T_{\alpha}\).

    For example for the quodratic case with \(\alpha = a + b \sqrt{d}\) and the
    standard basis \(\beta = \{ 1, \sqrt{d} \}\) \[\alpha \beta =
      \begin{pmatrix} a + b \sqrt{d} \\ a \sqrt{d} + bd \end{pmatrix}\] therefor
    \[T_{\alpha} = \begin{pmatrix} a && b \\ bd && a \end{pmatrix}.\]

    The norm and trace of \(\alpha \in L\) are defined respecitevely as
    \[N_{L / F}(\alpha) = \det(T_{\alpha}) \; and \; Tr_{L / F}(\alpha) =
      Tr(T_{\alpha}).\]
    Although \(T_{\alpha}\) depends on the choice of baises, norm and trace
    does not.

    (Trace of a matrix is the sum of its diagonal; for \(M_{nn} = (a_{ij})\)
    \(Tr(M) = \sum^{n}_{i} a_{ii}\))

    The map \(L \to M_{n}(F)\), \(\alpha \mapsto T_{\alpha}\) is an
    \textbf{injective ring homomorphism}. For \(f(x) \in F[x]\)
    \(T_{f(\alpha)} = f(T_{\alpha})\). \\
    This map implies many properties for norm and trace:
    \begin{itemize}
      \item \(N_{L / F}(\alpha^{n} + \beta^{m}) = \det(T_{\alpha}^{n} +
            T_{\beta}^{m})\)
      \item \(N_{L / F}(\alpha) = 0 \iff \alpha = 0\)
      \item \(N_{L / F}(\alpha \beta) = N_{L / F}(\alpha) N_{L / F}(\beta)\)
      \item For \(a \in F\) \(N_{L / F}(a) = a^{[L : F]}\) and
            \(Tr_{L / F}(a) = [L : F] a\)
      \item \(\forall a, b \in F\)
            \[Tr_{L / F}(a \alpha + b \beta) = a Tr_{L / F}(\alpha) +
            b Tr_{L / F}(\beta)\] meaning \(Tr_{L / F}\) is a F-linear map
    \end{itemize}


  \end{subsection}

  \begin{subsection}{Characteristic Polynomials}

    For \(A \in M_{n}(F)\), the characteristic polynomial of A is \(\chi_{A}(x)
    = \det(xI - A) \in F[x]\). \(\chi_{A}(x)\) is monic of degree n. For
    \(c_{i}\) coefficients of \(\chi_{A}(x)\), \(\det(A) = (-1){}^{n} c_{0}\)
    and \(Tr(A) = -c_{n - 1}\). \\
    For \(L / F\) size n, \(\alpha \in L\) we have \(T_{\alpha, L / F} \in
    M_{n}(F)\) with \(\chi_{T_{\alpha}} = \chi_{\alpha, L / F} =
    \chi_{\alpha}\) (same regardles of basis). This can be used to redefine
    \textbf{norm} and \textbf{trace} as \[N_{L / F}(\alpha) = (-1){}^{n} c_{0}
      \; and \; Tr_{L / F}(\alpha) = -c_{n - 1}.\] \\
    For a field extension \(L / F\) and \(\alpha \in L\) s.t.\
    \(L = F(\alpha)\), \(\chi_{\alpha}(x) = p_{\alpha}(x)\) with p the minimal
    polynomial. More generally for any field extension \(L / F\), \(n =
    [L : F]\) and \(d = [F(\alpha) : F]\) we have \(m = \frac{n}{d} =
    [L : F(\alpha)]\) and \(\chi_{\alpha}(x) = p_{\alpha, L / F}(x){}^{m}\). In
    the case above \(m = 1\). This yelds yet another definition of norm and
    trace, namely for \(a_{i}\) coefficients of \(p_{A}(x)\), \(\det(A) =
    (-1){}^{n} a_{0}^{m}\) and \(Tr(A) = -ma_{n - 1}\) with the same m.

  \end{subsection}

\end{section}

\begin{section}{Algebraic Stuff}

  \begin{subsection}{Number Fields}

    If \(\alpha \in \C\) is \textbf{algebraic} over \(\Q\), it
    is called an \textbf{algebraic number}. A field K s.t.\
    \(\Q \subseteq K \subseteq \C\) and \([K : \Q] <
    \infty\), is called an \textbf{algebraic number field} (or a number
    field). Every number field K is a \textbf{simple extension}, \(K =
    \Q(\theta)\) for some \(\theta \in K\). K finite implies K
    algebraic implies all elements algebraic. \\
    The set of all algebraic numbers is denoted as \(\bar{\Q}\) and is
    a field. Also \([\bar{\Q} : \Q]\) is infinite. For any
    \(\alpha \in \bar{\Q}\), the roots of \(p_{\alpha}(x)\) in
    \(\C\) are called \textbf{conjugates} of \(\alpha\). \\
    This leads to yet another definision of norm and trace, for K a number
    field, \(\alpha \in K\) and \(\alpha_{1}, \dots , \alpha_{n}\) (\(\alpha =
    \alpha_{i})\) conjugates of \(\alpha\), \(N_{K}(\alpha) = (\alpha_{1}
    \cdots \alpha_{n}){} ^{[K : \Q(\alpha)]}\) and \(Tr_{K}(\alpha) =
    [K : \Q(\alpha)](\alpha_{1} + \cdots + \alpha_{n})\).

  \end{subsection}

  \begin{subsection}{Integers}

    An \textbf{algebraic number} \(\alpha \in \bar{\Q}\) is said to be
    an \textbf{algebraic integer} if it is a root of a monic polynomial
    \(f(x) \in \Z[x]\). The set of all such integers is denoted as
    \(\bar{\Z}\). For K a number field, the set of all
    \textbf{algebraic integers} inside K is called \(\Ok_{K}\) and
    \(\alpha \in \Ok_{K}\) is called an \textbf{integer} in K. For
    \(\alpha \in \bar{\Q}\), the following statments are equivelant.
    \begin{itemize}
      \item \(\alpha \in \bar{\Z}\)
      \item \(p_{\alpha}(x) \in \Z\)
      \item \(\Z[\alpha]\) is generated by \(a_{i} \alpha^{i}\)
            for \(i \in [0, \deg p_{\alpha} - 1]\)
      \item There exists a \textbf{non trivial finitely generated abelian}
            group \((G, +)\), \(G \subset \C\) s.t.\ \(\alpha G
            \subseteq G\), meaing \(\forall g \in G \; \alpha g \in G\) \\
            (This group is typically but not always \(\Z[\alpha]\))
    \end{itemize}
    \(\bar{\Z}\) and for any number field K, \(\Ok_{K}\) are
    rings. For \(\alpha \in \Ok_{K}^{x}\) iff \(N_{K}(\alpha) =
    \pm 1\).

    For \(\alpha \in \bar{\Z}\) norm and trace of \(\alpha\) in K is
    in \(\Z\) for any K number field. \\
    Similarly to \(\Q\), any number field K can be generated by its
    integers by \(K = \{\frac{a}{m} \; | \; \alpha \in \Ok_{K}, m \in
    \Z, m \neq 0\}\).

  \end{subsection}

  \begin{subsection}{Quadratic Fields}

    \(d \in \Z\) is called \textbf{squarefree} if \(d \notin {0, 1}\)
    and no such prime p s.t.\ \(p^{2} | d\). Then for \(K / \Q\) with
    degree 2, \(K = \Q(\sqrt{d})\) for some \textbf{squarefree} d. Such
    fields are called \textbf{quadratic fields}, or \textbf{real quodratic
      fields} for \(d > 0\) and \textbf{imaginary quadratic fields} otherwise.

    For integers of such fields, there are two cases
    \[\Ok_{K} =
      \begin{cases}
        \Z[\frac{1 + \sqrt{d}}{2}] & d \equiv 1 \pmod{4} \\
        \Z[\sqrt{d}] & d \not\equiv 1 \pmod{4}
       \end{cases}
     \]

     From now on let \(K = \Q[\sqrt{d}]\), d not a square number,
     \(S = \Z[\sqrt{d}]\) or \(S = \Ok_{d}\) and
     \(\overline{a + b \sqrt{d}} = a - b \sqrt{d}\). \\
     \begin{itemize}
       \item \(d < -1 \Rightarrow \Z[\sqrt{d}]{}^{\times} =
             \{ \pm 1 \}\)
       \item \(d = -1 \Rightarrow \Ok_{d}^{\times} = \{ \pm 1, \pm
             i \}\)
       \item For \(d \equiv 1 \pmod{4}\) and \(d < -3\), we have
             \(\Ok_{d}^{\times} = \{ \pm 1 \}\)
       \item \(d = -3 \Rightarrow \Ok_{-3}^{\times} = \{ \pm 1, \pm
             \omega, \pm \omega^{2} \}\) where \(\omega = \frac{1 +
             \sqrt{-3}}{2}\)
       \item \(d > 1 \Rightarrow \exists u \in \Ok_{d}^{\times}\) s.t.\
             it is the smallest elements.
       \item \(\Ok_{K}^{\times} = \langle -1, u \rangle =
             \{ u^{r} | r \in \Z \}\)
     \end{itemize}
     u is called the \textbf{fundamental unit} of \(\Ok_{K}\).

  \end{subsection}

  \begin{subsection}{Real Quadratic Fields}

    For \(d > 1\) non square
    \begin{itemize}
      \item If \(S = \Z[\sqrt{d}]\) and \(a, b > 0\) s.t.\ \(a + b
            \sqrt{d} \in S\), it is \textbf{fundamental} if b is
            \textbf{minimal}
      \item If \(d \equiv 1 \pmod{4} \Rightarrow S =
            \Z[\frac{1 + \sqrt{d}}{2}]\)
            \begin{itemize}
              \item For \(d > 5\) for \(s, t > 0\) s.t.\ \(\frac{s + t
                    \sqrt{d}}{2}\), it is \textbf{fundamental} if t is
                    \textbf{minimal} (very similar to the
                    \(\Z[\sqrt{d}]\) case)
              \item For \(d = 5\), \(\frac{1 + \sqrt{5}}{2}\) is the
                    \textbf{fundamental} in \(\Ok_{5}\)
                    (this is useful because \(s, t = 1, 1\) as well as
                    \(s, t = 3, 1\) have t minimal so there would be ambiguity
                    in the above definision)
            \end{itemize}
    \end{itemize}

    For any \(\alpha = a + b \sqrt{d} \in \Q(d)\) \(\alpha >
    \sqrt{| N_{K}(\alpha) |} \iff a, b > 0\).

    \end{subsection}

  \begin{subsection}{Norm Equations}

    \begin{subsubsection}{Pell's Equation}

      \(x^{2} - dy^{2} = \pm 1\) where d not a square and \(x, y \in
      \Z\) is the \textbf{Pell's equation}. The LHS is the norm of
      \(x + y \sqrt{d}\), solving Pell's equation is the same as looking for
      elements of norm 1 in \(\Z[d]\). For a general version of the
      equation where RHS is any integer, solving it is equivelant to looking
      for all elements of \(\Z[d]\) with the norm equal to the RHS.

    The \textbf{Main theorem} can be used to find all solutions to such
    equations. For d \textbf{squarefree} and u the fundamental unit of
    \(\Z[d]\) the solutions are simply given by \[x = \pm
      \frac{u^{r} + \overline{u} ^{r}}{2} \; and \; y = \pm \frac{u^{r} -
        \overline{u}^{r}}{2 \sqrt{d}} \;\; \forall r \in \N.\] \\
    If d is not \textbf{squarefree} \(\Rightarrow a^{2} | d\) s.t.\
    \(\frac{d}{a^{2}}\) is \textbf{squarefree}, the solutions to x and
    \(z = ay\) are the same.
    (In that case u is taken to be the fundamental unit in \(\Z
    [\sqrt{\frac{d}{a^{2}}}]\))
    There is one other condition for \(z \in \Z[\sqrt{d}]\); namely
    \(a|z\). This is not always the case. To find the fundamental unit in
    \(\Z[\sqrt{d}]\) simply take the smallest power p of u s.t.\
    \(u^{p} \in \Z[\sqrt{d}]\). Then x, y are given by \[x = \pm
      \frac{u^{pr} + \bar{u}^{pr}}{2} \; and
      \; y = \pm \frac{u^{pr} - \bar{u}^{pr}}
      {2 \sqrt{d}} \;\; \forall r \in \N.\]

    \end{subsubsection}

  \end{subsection}

\end{section}

\begin{section}{Latticies}

  \begin{subsection}{Discriminants}

    The \textbf{discrimiant} of a number field is an integer that measures the
    \textit{size} of that number field (in some sense).

    \begin{subsubsection}{Basic Discriminant Definision}

      For \(f(x) \in \C[x]\) a monic polynomial degree \(n > 0\) with roots
      \(\theta_{1}, \dots , \theta_{n}\). The discrimiant of of \(f(x)\) is
      \[disc(f) = \prod_{1 \leq i < j \leq n}
        (\theta_{i} - \theta_{j}){}^{2}.\] (This is equal to 0 if \(f(x)\) has
      repeated  roots.) \\
      The discrimiant of the number field K s.t.\ \(\Ok_{K} = \Z[\theta]\) for
      some \(\theta \in K\) is the discriminant of the minimal polynomial of
      \(\theta\) over \(\Q\). It is denoted as \(\Delta_{K}\). \\
      \textit{This is not always appliciable as sometimes such \(\theta\) does
        not exist.}

    \end{subsubsection}

    \begin{subsubsection}{Discriminants of n-tuples}

      The \textbf{discriminant} of an n-tuple \(\gamma = (\gamma_{1}, \dots ,
      \gamma_{n}) \in K^{n}\) is \(\Delta_{K}(\gamma) := \det(Q(\gamma))\)
      where \(Q(\gamma) = (Tr_{K}(\gamma_{i} \gamma_{j})
      ){}_{1 \leq i < j \leq n}\), a n by n square symmetric matrix. \\
      This is in \(\Q\) and if all \(\gamma_{i} \in \Ok_{K}\), it is in \(\Z\).
      Also for some \(\delta = (\delta_{1}, \dots , \delta_{n})\) s.t.\
      \(\delta = M \gamma\) for a matrix \(M \in M_{n}(\Q)\), then
      \[\Delta_{K}(\delta) = \det(M){}^{2} \Delta_{K}(\gamma).\] This implies
      that the discrimiant is invariant under any permutation on the tuple.


      For a tuple \(\theta_{1}, \dots , \theta_{n} \in \C\), let
      \[ D =
        \begin{pmatrix}
          1 && \theta_{1} && \cdots && \theta^{n - 1}_{1} \\
          1 && \theta_{2} && \cdots && \theta^{n - 1}_{2} \\
          \vdots && \vdots && \cdots && \vdots \\
          1 && \theta_{n} && \cdots && \theta^{n - 1}_{n} \\
        \end{pmatrix}
      \]
      Then \(\det(D) = (-1){}^{{n}\choose{2}} \prod_{1 \leq r < s \leq n}
      (\theta_{r} - \theta_{s})\).

    \end{subsubsection}

    \begin{subsubsection}{Discriminants of Number Fields}

      For \(K = \Q(\theta)\) a number field with \(\theta_{1} \dots ,
      \theta_{n}\) denoting the roots of \(p_{\theta}(x)\), \(R = \Ok_{K}\) and
      \(\gamma = \{ 1, \theta, \dots , \theta^{n - 1}\}\). Then
      \(\Delta_{K}(\gamma) = (-1){}^{{n}\choose{2}} N_{K}(p'_{\theta}
      (\theta))\), where \(p'_{\theta}(x)\) is the derivative of
      \(p_{\theta}(x)\). \\
      If K is a quadratic field with \(\theta = \sqrt{d}\), d squarefree,
      \(\Delta_{K}(\{ 1, \theta \}) = 4d\). Note that if \(d \equiv 1 \pmod{4}\)
      \(\Delta_{K} \neq disc(p_{\theta})\). This is because in that case
      \(\{ 1, \theta \}\) does not generate R. In that case \(\Delta_{K}(
      \frac{1 + \theta}{2}) = d\).

      \(\Delta_{K}(\gamma) \neq 0 \iff \gamma\) is a \(\Q\)-baises of K.

      Generally \(\Delta_{K}\) only depends on the K, not on its generating
      basis. The discriminant of K is defined as \(\Delta_{K}(R) =
      \Delta_{K}(\alpha)\) with \(\alpha\) a generating basis.

    \end{subsubsection}

  \end{subsection}

  \begin{subsection}{Lattice Definision}

    Let A be a subgroup of a \(\Q\)-vector space V s.t.\ A is generated by a
    subset \(B \subseteq A\) which is a \(\Q\)-basis for V. Then A is called a
    \textbf{full lattice} in V and B is a \textbf{generating basis} for A.

    Every non-zero ideal \(I \subset R\) is a lattice in K.

  \end{subsection}

  \begin{subsection}{Basis}

    A generating basis for R is called \textbf{integral} basis for K. An
    integral basis needs to have \([K : \Q]\) elements. Such basis generates
    K with coefficients in \(\Q\) and generates R with coefficients in \(\Z\).

    For A, B full latitices in n dimentional V generated by \(\alpha, \beta\)
    and \(B \subset A\), there exists \(M \in M_{n}(\Z)\) s.t.
    \(\beta = M \alpha\) and
    \begin{itemize}
      \item \(| A / B | = | \det(M) |\)
      \item \(\Delta_{K}(\beta) = | A / B |^{2} \Delta_{K}(\alpha)
            \label{BAS}\)
    \end{itemize}

  \end{subsection}

  \begin{subsection}{Comparing Latticies}

    For \(S \subseteq R\) a full lattice in \(K = \Q(\theta)\) with generating
    basis \(\gamma = \{ \gamma_{1}, \dots , \gamma_{n}\}\) (for S). Let p be a
    prime s.t.\ p divides \(| R / S |\) then
    \begin{itemize}
      \item \(p^{2} | \Delta_{K}(S)\) (follows from~\ref{BAS})
      \item \(\exists \alpha = m_{1} \gamma_{1} + \cdots + m_{n} \gamma_{n}
            \in S\) with \(m_{i} \in \Z\) such that \(\frac{\alpha}{p} \in R\)
            and \(\notin S\). Also \(p = 2 \Rightarrow 0 \leq m_{i} \leq 1\)
            and otherwise \(0 \leq | m_{i} | < \frac{p}{2}\). (Note: \(\alpha
            \neq 0\))
    \end{itemize}
    This can be used to show that \(S = R\) as if no such p can exist they have
    to equall. Morever \(S = \Z[\theta]\) can be used to check if
    \(\Ok = \Z[\theta]\). This is done by taking an element \(\alpha \in S\)
    s.t. \(\frac{\alpha}{p}\) with coefficients \(| a_{i} | < \frac{p}{2}\).

  \end{subsection}

\end{section}

\begin{section}{Rings of Integers}

  From now on \(R = \Ok_{K}\) for some number field K.

  \begin{subsection}{Basic Group Structure}

    As unique factorisation of elements in R often fails, exploring
    factorisation of prime ideals of R is a good idea as it is always unique.

    \(\mathcal{I}(R)\) is an abelian group.

    \begin{subsubsection}{Operation}

      The product of two ideals I, J of R is defined as
      \[IJ = \{ \sum^{k}_{i = i} x_{i}y_{i} | k \in \N, x_{i} \in I,
        y_{i} \in J \}.\] This is still an ideal of R. Note that \(IJ \subseteq
      I\), meaning the product decreases the size of the ideal. Divisibility
      \(I | J\) can be defined as \(\exists S\) s.t.\ \(J = IS\). If S
      exists (meaning I divides J), \(J = IS \subseteq I\), meaning that
      \textit{to divide is to contain}. The reverse is also true;
      \textit{to contain is to divide}.

      This operation is commutative and associative. Morever for generators
      \(I = (a_{1}, \dots , a_{m})\) and \(J = (b_{1}, \dots , b_{n})\) then
      \(IJ = (a_{i}b_{j} | 1 \leq i \leq m, 1 \leq j \leq n)\).

      The ideal R behaves like the identity.

    \end{subsubsection}

    \begin{subsubsection}{Primes}

      Let \(\p\) be a prime ideal in R. Then \[\p \supseteq IJ \iff
        \p \supseteq I \;\; or \;\; \p \supseteq J.\]

      All ideals can be \textbf{uniquely} expressed in terms of prime ideals.

    \end{subsubsection}

    \begin{subsubsection}{Inverses}

      A fractional ideal of R is a subset of K of the form \[\lambda I =
        \{ \lambda a | a \in I\},\] where I is a non zero ideal of R and
      \(\lambda \in K^{\times}\). \(\lambda R\) is called \textbf{principal
        fractional ideal}. The set of all fractional ideals of R is denoted
      as \(\mathcal{I}(R)\) and the set of \textbf{principal} ones is
      \(\mathcal{P}(R)\).

      Every ideal is a fractional ideal (by taking \(\lambda = 1\)) but the
      way may not hold as \(\lambda I\) can be not in R. But for every
      fractional ideal \(\exists x \in R\) s.t.\ \(x \lambda I \subseteq R\).

      The same operation as defined for normal ideals holds fractional ideals.

      A fractional ideal \(\id{a}\) is called \textbf{invertible} if \(\exists
      \id{b}\) s.t.\ \(\id{a} \id{b} = R\). Then \(\id{b} = \id{a}^{-1}\). \\
      \textbf{Ever fractional ideal is \emph{invertible}.}

    \end{subsubsection}

    \begin{subsubsection}{Divisibily}

      For \(J \subseteq I\) ideals in R
      \begin{itemize}
        \item \(I^{-1} J\) is an ideal in R, therefore \(I | J\)
        \item \(I \subseteq I^{-1} J\)
      \end{itemize}

    \end{subsubsection}

    \begin{subsubsection}{Conjugation in Quodratics}

      For any element \(\alpha \in K\), there exists a unique \(\bar{\alpha}
      \in K\) (\(\alpha \in \Q \iff \bar{\alpha} \in \Q\)). Similarly for any
      ideal I, \(\bar{I} = \{ \bar{\alpha} | \alpha \in I \}\). This is a proper
      ideal of R. Morever if \(I = (\alpha, \beta)_{R}\); \(\bar{I} =
      (\bar{\alpha}, \bar{\beta})_{R}\). \\
      \(I\) prime \(\iff \bar{I}\) prime and \(N(I) = N(\bar{I})\).

    \end{subsubsection}

  \end{subsection}

  \begin{subsection}{Norm}

    \begin{subsubsection}{Definision}

      For I a non-zero ideal of R, \(| R / I | \leq \infty\) and that size
      defines its norm denoted \(N(I)\).
      \begin{itemize}
        \item \(N(I) \in \N\)
        \item \(N(R) = 1\)
        \item \(J \subseteq I \iff N(I) \leq N(J)\) and
              \(J \subset I \iff N(I) < N(J)\)
      \end{itemize}
      For a principal ideal \(I = (\alpha)_{R}\), \(N(I) = N(\alpha) =
      | N_{K}(\alpha) |\). \\
      For an ideal \(I = (\alpha, \beta)_{R}\), \(N(I) = \gcd(
      \alpha \bar{\alpha}, \beta \bar{\beta}, \alpha \bar{\beta} + \bar{\alpha} \beta)\).

    \end{subsubsection}

    \begin{subsubsection}{Properties}

      The norm decreases as the size of the ideal increases.

      For an ideal \(I = (\alpha)_{R}\) \(N(I) = Nr_{K}(\alpha)\).

      \(N(IJ) = N(I)N(J)\)

    \end{subsubsection}

    \begin{subsubsection}{Calculations}

      For \(\p\) a prime non zero ideal: \\
      \(\p | (N(\p))_{R}\). This implies \((N(\p)){}_{R} \subset \p\) and
      \(N(\p) \in \p\). Also \(\p \cap \Z \neq (0)\). In fact the ideal
      \(\p \cap \Z = (p)_{\Z}\) which is a non-zero prime ideal. \\
      It can be said that \(\p\) \textit{lies above} p, which is equivalant to
      \(\p | (p)_{R}\). This is true iff \(N(\p) = p^{r}\) for
      \(1 \leq r \leq n\) and \(n = [K : \Q]\).

    \end{subsubsection}

  \end{subsection}

  \begin{subsection}{Unique Factorisation}

    As mentioned above, any non-zero ideal of R has a uniqe factorisation to
    prime ideals. \\
    Rings with this property are called \textbf{Dedekind Domains}.

    If a ring of integers \(\Ok_{k} = R\) is a UFD, it is also a PID (the
    converse is always true). R is a UFD iff \(Cl(R) = \{ e \}\) (is trivial).

    Every non-zero prime ideal of R is \textbf{maximal} (not true in general,
    but the converse is).

    \begin{subsubsection}{Factorisation}

      For an ideal \(I = (\alpha_{1}, \alpha_{2}, \dots , \alpha_{n})\)
      factorise all \(J_{i} = \alpha_{i}\) by calculating the norm. Then
      \(I = \gcd(J_{1}, \dots , J_{n})\).

    \end{subsubsection}

  \end{subsection}

  \begin{subsection}{Spitting of Primes}

    \begin{subsubsection}{The Theorem}

      \textit{Important note: \(x^{n}\) factorises to \((x)(x) \dots\)}

      For \(R = \Ok_{K} = \Z[\theta]\) for some \(\theta \in K\), \(p \in \Z\) a
      prime and \(\bar{p}_{\theta}(x) = \bar{f}_{1}(x){}^{m_{1}} \cdots
      \bar{f}_{r}(x)^{m_{r}}\) where \(\bar{p}_{\theta}, \bar{f}_{i} \in
      \F_{p}[x]\) and \(m_{i} \in \N\); \[(p)_{R} = \p^{m_{1}}_{1} \cdots
        \p^{m_{r}}_{r}\] with \[\p_{i} = (p, \bar{f}_{i}(\theta)){}_{R}.\]
      All \(\p_{i}s\) have the norm of \(N(\p_{i}) = p^{deg(f_{i})}\).

    \end{subsubsection}

    \begin{subsubsection}{Quadratics}

      If \(\p\) \textbf{lies above} p, \(\p \bar{\p} = (N(\p))_{R}\).

      This always holds for \textbf{quodratic} number fields. Furthermore there
      are only three mutally exclusive options for a prime p
      \begin{itemize}
        \item \textbf{Inert}: \((p)_{R}\) is a prime ideal,
              \(\p = \bar{\p}\) and \(N(\p) = p^{2}\)
        \item \textbf{Split}: \((p)_{R} = \p \bar{\p}\),
              \(\p \neq \bar{\p}\) and \(N(\p) = N(\bar{\p}) = p\)
        \item \textbf{Ramified}: \((p)_{R} = \p^{2}\),
              \(\p = \bar{\p}\) and \(N(\p) = p\)
              (More generally this is the name if any of the \(m_{i} \geq 2\),
              this is only true if \(p | \delta_{K}\))
      \end{itemize}

    \end{subsubsection}

  \end{subsection}

\end{section}

\begin{section}{The Ideal Class}

  The quotient \(Cl(R) := \mathcal{I}(R) / \mathcal{P}(R)\). It is called the
  \textbf{ideal class group} of R. Then for every ideal \(I \in
  \mathcal{I}(R)\), \([I]\) denoteds the class \(I \times \mathcal{P}(R)\). \\
  Clearly \([\p]^{-1} = [\bar{\p}]\).

  \begin{subsection}{Finitnes and Order}

    For any given norm \(n \in \N\), \(|\{ N(I) < n | I \in R \}|\) is finite.
    More importantly, \(Cl(R)\) is finite. Its order is called the \textbf{class
      number} of K, denoted \(h_{k}\) (or h).

    \begin{subsubsection}{Minkowski Bound}

      The upper bound of the order of the class goup dependant on the
      discriminant.

      For a number field \(K = \Q(\theta)\) of degree \(n = s + 2t\), where s is
      the number of real roots and t is the number of pairs of complex roots,
      \[B_{K} = (\frac{4}{\pi})^{t} \frac{n!}{n^{n}} \sqrt{| \Delta_{K} |}.\]
      Then for each class \([I] \in Cl(R)\), there exits J a non fractional
      ideal s.t.\ \[N(J) \leq B_{K}.\]

    \end{subsubsection}

  \end{subsection}

\end{section}

\begin{section}{Equations}

  \begin{subsection}{Mordell}

    \begin{subsubsection}{Base Case}

      For an equation \[x^{2} - s^{2} d = y^{3},\] with d sqrarefree check
      \(K = \Q(\sqrt{d})\). If \(h_{K} = 1\), it is a UFD factorise into
      \((x + s \sqrt{d})(x - s \sqrt{d}) \in K\). Then both terms have to be
      cubes, therefor \(x + s \sqrt{d} = (u + v \sqrt{d})^{3}\) for some
      \(u, v \in \Z\).

    \end{subsubsection}

    \begin{subsubsection}{Class Number 2}

      With \(h_{K} = 2\) it is harder but a similar process can be applied with
      ideals. First take \(\alpha = x + s \sqrt{d}\) so
      \(N_{K}(\alpha) = y^{3}\). Then find all ideals \(\p\) s.t.
      \(\p|(\alpha - \bar{\alpha})\) and write \((\alpha) = \p Q\) and
      \(\bar{\alpha} = \p \bar{Q}\) for \(Q\) and \(\bar{Q}\) cooprime to all
      \(\p\) and to each other. \\
      From there write \((y^{3}) = \p^{2} Q \bar{Q}\). Due to uniqe
      factorisation of ideals this means that \((\alpha) = I^{3}\) for some
      ideal \(I\). As \(h_{k} = 2\); \([I^{3}] = e \Rightarrow [I] = e\)
      and \[\exists \beta \;\; s.t. \;\; (\beta^{3}) = (\alpha) \Rightarrow \beta^{3} = \alpha,\]
      getting the same enaquolity as for \(h_{K} = 1\) case.

      For \(\gcd(h_{K}, 3) \neq 1\) find a prime that divides y but not x (can
      be achived using modulo reductions, esspecially viable if d is odd, then
      try 2). Then for \(h_{K} = 3\); \(\p = (\alpha / p) (I^{3})^{-1}\) for some I where
      the RHS is principle as all cubes are. This can be used to find a solution
      or show there is none depening on the ideal \(\p\).

    \end{subsubsection}

  \end{subsection}

  \begin{subsection}{Pell 2.0}

    An equation \[x^{2} - s^{2} d y^{2} = \pm a\] with d squarefree can be
    seolved similar method similar to the one earlier can be used. The only
    difference is the a rather than 1. This can be achived by finding an element
    \(\beta \in K = \Q(\sqrt{d})\) s.t.\ \(N_{K}(\beta) = a\). Then the
    solutions are
    \[x = \pm \frac{u^{pr} \beta + \bar{u}^{pr} \bar{\beta}}{2} \; and
      \; y = \pm \frac{u^{pr} \beta - \bar{u}^{pr} \bar{\beta}}
      {s \times 2 \sqrt{d}} \;\; \forall r \in \N.\]

  \end{subsection}

\end{section}

\newpage

\begin{section}{P-adic Numbers (Extra Reading)}

  \begin{subsection}{Motivation}

    The main motivation is solving \(x^{2} \equiv a \pmod{p^{n}}\) for an odd
    prime p and a coprime to p. A solution \(x \in \Z\) exists for \(n = 1\) iff
    \((\frac{a}{p}) = 1\). In that case take \(b^{2}_{0} \equiv a \pmod{p}\).
    The claim is that this implies solutions exists \(\forall n \in \N\). \\
    Asssuming a solution x exists for \(n \geq 1\), then x is coprime to p, so
    \(x_{1} \equiv \frac{1}{2} (x + \frac{a}{x}) \pmod{p^{2n}}\) also exists
    (This is the standard \textbf{Newton-Raphson iterative method}). Then \\
    \(x_{1} - x \equiv 0 \pmod{p^{n}}\), and \(x^{2}_{1} - a \equiv 0
    \pmod{p^{2n}}\).

    This prcedure can be used to get \(x_{0}, x_{1}, \dots , x_{n}\) s.t.\
    \(x_{k} \equiv a^{2} \pmod{p^{2^{k}}}\) and \(x_{k + 1} \equiv x_{k}
    \pmod{p^{2^{k}}}\). \\
    These xs can be writen in base p s.t.\
    \[x_{0} = b_{0}\] \[x_{1} = b_{0} + b_{1} p\]
    \[x_{2} = b_{0} + b_{1} p + b_{2} p^{2} + b_{3} p^{3}\]
    and so on, all \(\pmod{p^{2^{k}}}\). In general
    \[x_{k} = \sum^{2^{k}}_{i = 0} b_{i}p^{i} \pmod{p^{2^{k}}}.\]
    From this it follows that \(x_{\infty} = \pdc[b]\)
    is a root of \(x^{2} \equiv a \pmod{p^{\infty}}\) in a field \(\Q_{p}\) of
    padic numbers.

  \end{subsection}

  \begin{subsection}{Valuations}

    A valuation \(\val\) on a field \(\F\) is a map \(\F \mapsto \R^{+}\)
    satisfying for each \(x, y \in \F\)
    \begin{align*}
      \val[x] = 0 \iff x = 0 \tag{ZERo} \label{ZER} \\
      \val[xy] = \val[x] \cdot \val[y] \tag{HOMomorphism} \label{HOM} \\
      \val[x + y] \leq \val[x] + \val[y] \tag{TRIangle} \label{TRI}
    \end{align*}
    If it also satisfies \[\val[x + y] \leq \max(\val[x], \val[y]) \label{MAX}
      \tag{MAXimum},\] it is called a \textbf{nonarchimedean} valuation.
    Otherwise it is called \textbf{archimedean}. If~\ref{MAX} holds,~\ref{TRI}
    certainly holds.

    For any field \(\F\) and any \(n, m \in \N\) where \(\N\) is defined as a
    sum of copies of the identity in \(\F\), \(\val[n] = \val[-n]\) and
    \(\frac{\val[m]}{\val[n]} = \val[\frac{m}{n}]\).

    \begin{subsubsection}{Nonarchimedean Valuations}

      For a \textbf{nonarchimedean} valuation and \(x, y \in \F\) s.t.\ \(x
      \neq y\), \(\val[ x + y] = \max(x, y)\). Similarly for \(\val[ x_{1} +
      \cdots + x_{n}] \leq \max(\val[x_{1} , \dots , x_{n}])\) with equality if the
      maximum is unique. This also implies that for \(n \in \N\), \(\val[n] \leq
      1\). \\
      For \(| \cdot |\) a \textbf{nonarchimedean} valuation, so is \(\val
      ^{\alpha}\) for any \(0 < \alpha \in \R\).

      If \(\val\) is \textbf{nonarchimedean} valuation on \(\Q\) with some
      \(\val[n] < 1\) for some \(n \in \N\), then there is a prime p s.t.\
      \(\forall n \in \N\) if \(\val[n] < 1 \iff p | n\).

    \end{subsubsection}

    \begin{subsubsection}{P-adic Valuations}

      There is a nonarchimedean valuation corresponding to each prime p. Define
      \(\val_{p}\) by \(\val[0]{}_{p} = 0\) and \(\val[\frac{p^{k}n}{m}] =
      p^{-k}\) for n, m coprime to p. This impiles that \(\val[n]{}_{p} = 1\)
      if n is coprime to p as well as \(\val[p]{}_{p} = \frac{1}{p}\). This is
      called a p-adic valuation on \(\Q\). \textit{Note: the choice of
        \(p^{-k}\) is not important as we can take any \(\alpha\) to make this
        equal to any number in range \((0, 1)\).}

    \end{subsubsection}

  \end{subsection}

  \begin{subsection}{Defining the Field}

    \begin{subsubsection}{Reals (Revision)}

      \textit{Recal from Real Analysis}. The real numbers \(\R\) are defined by
      equivlancy classes of \textbf{Cauchy sequences}. A sequence \((a_{n}) =
      a_{1}, a_{2}, \dots , a_{n} , \dots\) for all \(a_{n} \in \Q\) is Cauchy
      iff for each \(\epsilon > 0 \; \exists N > 0\) s.t.\ \(\val[a_{n} -
      a_{m}] < \epsilon \; \forall n, m > N\). Then two sequences \((a_{n}),
      (b_{n})\) are equivelant if the sequence \(a_{1}, b_{1}, \dots , a_{n}
      b_{n}, \dots\) is also Cauchy. This is for \(\val\) being defined as
      absolute value.

    \end{subsubsection}

    \begin{subsubsection}{\(\Q_{p}\)}

      A field of p-adic numbers \(\Q_{p}\) as a completion of \(\Q\) replacing
      the absolute value with \(\val_{p}\) and obtain p-Cauchy sequences this
      way.

      Any rational number \(\frac{m}{n}\) s.t.\ \(\val[\frac{m}{n}]{}_{p} = 1\)
      can be approximated arbitrarily closely by a positive integer. Formally
      for any \(k \in \N \; \exists N \in \N\) s.t.\ \(\val[\frac{m}{n} - N]{}
      _{p} \leq p^{-k}\). This essentially removes the need for fractoins and
      decimal points to express \(\Q\). This result can be easily extended to
      fractions not coprime to p. \\
      This positive integer N can be writen in base p, \[p^{k}N = p^{k}
        (a_{0} + a_{1} p + \cdots + a_{r} p^{r})\] where all \(a_{i}\) are
      integers mod p (assume \(a_{0} \neq 0\)).

      \(\Q_{p}\) is then defined to be the set o all equivelance classes of
      p-Cauchy sequences. Every such class can be represented as N above with
      r at \(\infty\); namely \(\pdc\). Then we define addition and
      multiplication as (...)

      Under these operation \(\Q_{p}\) is an infinite, uncountable field of
      characteristic 0 (similarly to \(\R\)). The p-adic integers are defined
      as \(a \in \Z_{p} \iff \val[a]{}_{p} \leq 1\) (so if \(a \in \Z\),
      \(a \in \Z_{p}\)). \(\Z_{p}\) is a ring.
      \textit{Add units when confirmed}.

    \end{subsubsection}

    \begin{subsubsection}{Basic Operations}

      For this section \(a = \pdc\).

      \textbf{Negation}: \\
      \(-a = p^{k}((p - a_{0}) + \pdc[-1 - a][k][1])\)

      \textbf{Reciprocals}: \\
      \(a^{-1} = \pdc[a'][-k]\) where \(a'_{i}s\) can be found by computing
      \(N = a_{0} + \cdots , a_{i} p^{i}\), then calculating \(p^{i + 1} > N'
      \in \N\) s.t.\ \(NN' \equiv 1 \pmod{p^{i + 1}}\). Then writing N' in base
      p gives \(a'_{i}s\).

      \textbf{Addition}: \\
      For \(b = \pdc[b][m]\) and \(m = k\), \(a + b = p^{k}((a_{0} + b_{0}) +
      \cdots + (a_{i} + b_{i})p^{i} + \cdots)\) (all terms modulo p). If \(k >
      m\) shift \(b_{i}s\) to \(b_{i + k - m}\) and add 0s to the front of the
      sequence.

      \textbf{Multiplication}: \\
      \(a * b = p^{k + m}(a_{0}b_{0} + (a_{1}b_{0} + a_{0}b_{1})p + \cdots +
      (\sum^{i}_{j = 0} a_{j} b_{i - j})p^{i})\) (all terms modulo p).

    \end{subsubsection}

    \begin{subsubsection}{Representation}

      Some complex stuff

    \end{subsubsection}

  \end{subsection}

\end{section}

\end{document}
