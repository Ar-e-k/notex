\documentclass[12pt, letterpaper]{article}

\usepackage{graphicx}
\usepackage{parskip} % Disabling paragraph index as it does not fit maths
\usepackage{hyperref} % Usable menu and references
\usepackage{amssymb} % Used to show sets of sumbers, like the real numbers
\usepackage{amsmath} % Used for collumn vectors

\graphicspath{{images}}

\title{Number Theory}
\author{Arkadiusz Naks}
\date{2023}

\begin{document}

\begin{titlepage}
  \begin{center}
    \makeatletter
    \vspace*{1cm}
    \Huge
    \textbf{\@title}

    \vspace{0.5cm}
    \Large
    Lecture notes from Number Theory at Durham University

    \vspace{1.5cm}

    \textbf{\@author}

    % \includegraphics[scale=0.55]{.png}
    \vfill

    \vspace{0.8cm}

    \small
    Based on my understanding of lectures and notes of \\
    \@date{}
  \end{center}
\end{titlepage}

\tableofcontents
\newpage

\begin{section}{Important Definisions}

  A place for short and important definisions \\

  \textsc{Definision} (Diophantine equation) \textit{Polynomial equations with
    integer coefficients}

  \textsc{Definision} (Regular Prime) \textit{A prime p which does not difide
    the order of the ideal class group of \(\mathbb{Q}(\xi_{p})\) with \(\xi_{p}\)
    being the pth root of unity}

\end{section}

\begin{section}{Introduction}

  Most classic example of a question solved through number theory is the famous
  Fermat's Last Theorem (FLT) (actually a conjecture), which states that
  \[x^{n} + y^{n} = z^{n}\] has no solutions in \(\mathbb{Z}\) excluding 0 when
  \(n > 0\). The proof is very complex and therefor omited. One aim of varphis
  course will be to solve \textbf{diophantine} equations with integer or
  rational solutions.

\end{section}

\begin{section}{Divisibility}

  \begin{subsection}{Euclidean Domains}

    Every ED is a \textbf{principal ideal domain}. \\
    An \textbf{euclidean domain} is a ring R which has a
    \textbf{euclidean function} \(\varphi\). This function is defined
    as \(\varphi: R \backslash{} \{0\} \to \mathbb{N}_{0}\) such that:
    \begin{itemize}
      \item \(\forall x, y \in R \backslash{} \{0\}\) \(\varphi(x) \leq \varphi(xy)\)
      \item \(\forall x \in R, y \in R \backslash{} \{0\} \exists q, r\) s.t.
            \(x = qy + r\) with \(\varphi(r) < \varphi(y)\)
    \end{itemize}

    Some examples of EDs are:
    \begin{itemize}
      \item \(\mathbb{Z}\) with \(\varphi: x \to |x|\)
      \item \(F[x]\) where F is a field with \(varphi: f(x) \to \deg(f)\)
      \item \(\mathbb{Z}[\sqrt{-2}]\), with \(\varphi: a + b\sqrt{2}
            \to a^{2} + 2b^{3}\)
    \end{itemize}

  \end{subsection}

\end{section}

\begin{section}{Field Extensions}

  \begin{subsection}{Basic Definision}

    For fields F and L, if F is contained in L and they have the same operations,
    F is called a \textbf{subfield} of L and L is a \textbf{field extention} of
    F, denoted as \(L/F\). Every \textbf{field extension} \(L/F\) implies L is a
    vectorspace over F, with usual vectorspace properties
    \begin{itemize}
      \item \(0 \in L\)
      \item \(u, v \in L \to u + v \in L\)
      \item \(a \in F\) and \(u \in L \to au \in L\)
    \end{itemize}
    (Unsure why this is useful as \(a \in F \to a \in L\) and a field is closed
    under both addition and multiplication)

    Each field extension \(L/F\) has a \textbf{degree} \([L : F]\) which s the
    dimension \(\dim_{F}L\) and can be infinite. If the dimension is finite, \(L/F\)
    is called a \textbf{finite field extension}.

  \end{subsection}

  \begin{subsection}{Algebraic Field Extensions}

    For a field extension \(L/F\), any element \(\alpha\in L\) is called
    \textbf{algebraic over F} if \(\exists f(x) \in F[x], f(x) \neq 0\) s.t.
    \(f(\alpha) = 0\). If all elements of L are algebraic over F, \(L/F\) is
    called an \textbf{algebraic extension} (alternitevly it is said that L is
    algebraic over F). \\
    \textit{Only a finite dimention field extension can be algebraic.} \\
    In fact all \textbf{finite} field extensions are \textbf{algebraic}.

    For \(\alpha \in L\) algebraic, its \textbf{minimal polynomial}
    \(p_{\alpha}(x)\) (or more precisely \(p_{\alpha, F}(x)\) when the field is
    not clear) is defined to be the monic polynomial of smallest degree s.t.\
    \(\alpha\) is the root. This always exists as \(\alpha\) is algebraic. It is
    also always unique. This polynomial is useful as its coefficients \(\in F\)
    containe information about \(\alpha \in L\). The polynomial is also always
    irreducable over F. \\
    Morever if \(f(\alpha) = 0\), f is monic and irreducable, it is the
    \textbf{minimal polynomial} of \(\alpha\). \\
    \textit{The \textbf{degree} of \(\alpha\) is defined as the \textbf{degree}
      of its minimal polynomial}.

  \end{subsection}

  \begin{subsection}{Generated Fields}

    For a \textbf{field extension} \(L / F\) and \(\alpha \in L\) we define
    \(F(\alpha) \subseteq L\) to be the smallest field extension of F which
    contains \(\alpha\); \(F(\alpha) = K\) where K is a field, \(\alpha \in K\)
    and \(F \subseteq K \subseteq L\) (K is the smalles subfield of L which
    contains both F and \(\alpha\)). Note \(\alpha \notin F\). \(F(\alpha)\)
    is called the \textbf{field generated by \(\alpha\) over F}. \\
    More generally \(F(\alpha_{1}, \dots , \alpha_{n}) = F(\alpha_{1})
    (\alpha_{2}) \dots F(\alpha_{n})\). \\
    It is important to remember that generally \(F(\alpha) \neq F[\alpha]\) as
    if \(\alpha\) is not \textbf{algebraic} (over F), \(\alpha^{-1} \notin
    F[\alpha]\) and \(F[\alpha]\) is not a field. On the other hand, if
    \(\alpha\) is \textbf{algebraic}, \(F[\alpha] = F(\alpha)\). In such case,
    the degree of \(F(\alpha) / F\) is the same as the degree of \(\alpha\),
    \([F(\alpha) : F]= \deg p_{\alpha, F}\). \\
    If \(K / F\) and \(L / K\) are field extensions, s.t. \(F \subseteq K
    \subseteq L\), then it is called a \textbf{tower} of fields and
    \([L : F] = [L : K] \times [K : F]\) (if one side is infinite, so is the
    other).

  \end{subsection}

  \begin{subsection}{Norm and Trace}

    Let \(L / F\) finite field extension and \(n = [L : F]\). Then as L is a
    vector space over F for any \(\alpha \in L\) there is a F-linear map
    \[\hat \alpha: L \to L, \; x \mapsto \alpha x.\]
    Then for a basis of L over F \((\beta_{i})_{i \in [1, n]}\) and a
    \(n \times n\) matrix \(T_{\alpha} = T_{\alpha, L / F} \in M_{n}(F)\) is
    defined to be \[\hat \alpha(\beta_{i}) = \alpha \beta_{i} =
      t_{i1} \beta_{i} + \cdots + t_{in} \beta_{i}\] where \(t_{ij} \in F\) and
    \(T_{\alpha} = (t_{ij})\). This means that
    \[\alpha \begin{pmatrix} \beta_{1} \\ \vdots \\ \beta_{i} \end{pmatrix} =
      T_{\alpha} \begin{pmatrix} \beta_{1} \\ \vdots \\ \beta_{i} \end{pmatrix}
      ,\]
    and therefore \(\alpha\) is an \textbf{eigenvalue} of the matrix
    \(T_{\alpha}\).

    The norm and trace of \(\alpha \in L\) are defined respecitevely as
    \[N_{L / F}(\alpha) = \det(T_{\alpha}) \; and \; Tr_{L / F}(\alpha) =
      Tr(T_{\alpha}).\]
    Although \(T_{\alpha}\) depends on the choice of baises, norm and trace
    does not.

    (Trace of a matrix is the sum of its diagonal; for \(M_{nn} = (a_{ij})\)
    \(Tr(M) = \sum^{n}_{i} a_{ii}\))

    The map \(L \to M_{n}(F)\), \(\alpha \mapsto T_{\alpha}\) is an
    \textbf{injective ring homomorphism}. For \(f(x) \in F[x]\)
    \(T_{f(\alpha)} = f(T_{\alpha})\). \\
    This map implies many properties for norm and trace:
    \begin{itemize}
      \item \(N_{L / F}(\alpha^{n} + \beta^{m}) = \det(T_{\alpha}^{n} +
            T_{\beta}^{m})\)
      \item \(N_{L / F}(\alpha) = 0 \iff \alpha = 0\)
      \item \(N_{L / F}(\alpha \beta) = N_{L / F}(\alpha) N_{L / F}(\beta)\)
      \item For \(a \in F\) \(N_{L / F} = a^{[L : F]}\) and
            \(Tr_{L / F}(a) = [L : F] a\)
      \item \(\forall a, b \in F\)
            \[Tr_{L / F}(a \alpha + b \beta) = a Tr_{L / F}(\alpha) +
            b Tr_{L / F}(\beta)\] meaning \(Tr_{L / F}\) is a F-linear map
    \end{itemize}


  \end{subsection}

  \begin{subsection}{Characteristic Polynomials}

    For \(A \in M_{n}(F)\) the characteristic polynomial of A, \(\chi_{A}(x) =
    \det(xI - A) \in F[x]\). \(\chi_{A}(x)\) is monic of degree n. For \(c_{i}\)
    coefficients of \(\chi_{A}(x)\), \(\det(A) = (-1){}^{n} c_{0}\) and
    \(Tr(A) = -c_{n - 1}\). \\
    For \(L / F\) size n, \(\alpha \in L\) we have \(T_{\alpha, L / F} \in
    M_{n}(F)\) with \(\chi_{T_{\alpha}} = \chi_{\alpha, L / F} =
    \chi_{\alpha}\) (same regardles of basis). This can be used to redefine
    \textbf{norm} and \textbf{trace} as \[N_{L / F}(\alpha) = (-1){}^{n} c_{0}
      \; and \; Tr_{L / F}(\alpha) = -c_{n - 1}.\] \\
    For a field extension \(L / F\) and \(\alpha \in L\) s.t.\
    \(L = F(\alpha)\), \(\chi_{\alpha}(x) = p_{\alpha}(x)\) with p the minimal
    polynomial. More generally for any field extension \(L / F\), \(n =
    [L : K]\)and \(d = [F(\alpha) : F]\) we have \(m = \frac{n}{d} =
    [L : F(\alpha)]\) and \(\chi_{\alpha}(x) = p_{\alpha}(x){}^{m}\). In the
    case above \(m = 1\). This yelds yet another definition of norm and trace,
    namely for \(a_{i}\) coefficients of \(p_{A}(x)\), \(\det(A) = (-1){}^{n}
    a_{0}^{m}\) and \(Tr(A) = ma_{n - 1}\) with the same m.

  \end{subsection}

\end{section}

\end{document}
