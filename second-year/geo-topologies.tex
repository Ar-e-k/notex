\documentclass[12pt, letterpaper]{article}

\usepackage{graphicx}
\usepackage{mathtools}
\usepackage{tikz} % Used for drawing arcs for knots
\usepackage{parskip} % Disabling paragraph index as it does not fit maths
\usepackage{amssymb} % Used to show sets of sumbers, like the real numbers

% Copied from, with some minor changes and major additions:
% https://tex.stackexchange.com/questions/306004/drawing-the-rules-for-the-bracket-polynomial
\newcommand{\KP}[1]{%
  \begin{tikzpicture}[baseline=-\dimexpr\fontdimen22\textfont2\relax]
  #1
  \end{tikzpicture}%
}
\newcommand{\Circle}{%
  \KP{\filldraw[color=gray, fill=none, thick] circle (0.3);}%
}
\newcommand{\UCross}{%
  \KP{
    \draw[color=gray,thick] (-0.3,0.3) -- (0.3,-0.3);
    \draw[color=gray,thick] (-0.3,-0.3) -- (-0.05,-0.05);
    \draw[color=gray,thick] (0.05,0.05) -- (0.3,0.3);
  }%
}
\newcommand{\UOCross}{%
  \KP{
    \draw[color=gray,thick] (-0.3,0.3) -- (-0.05,0.05);
    \draw[color=gray,thick] (0.05,-0.05) -- (0.3,-0.3);
    \draw[color=gray,thick] (-0.3,-0.3) -- (0.3,0.3);
  }%
}
\newcommand{\DPCross}{%
  \KP{
    \draw[color=gray,thick,->] (-0.3,0.3) -- (0.3,-0.3);
    \draw[color=gray,thick] (-0.3,-0.3) -- (-0.05,-0.05);
    \draw[color=gray,thick,->] (0.05,0.05) -- (0.3,0.3);
  }%
}
\newcommand{\DNCross}{%
  \KP{
    \draw[color=gray,thick] (-0.3,0.3) -- (-0.05,0.05);
    \draw[color=gray,thick,->] (0.05,-0.05) -- (0.3,-0.3);
    \draw[color=gray,thick,->] (-0.3,-0.3) -- (0.3,0.3);
  }%
}
\newcommand{\RSmooth}{%
  \KP{%
    \draw[color=gray,thick] (-0.3,0.3)..controls (0,-0.05).. (0.3,0.3);
    \draw[color=gray,thick] (-0.3,-0.3)..controls (0,0.05).. (0.3,-0.3);
  }%
}
\newcommand{\LSmooth}{%
  \KP{%
    \draw[color=gray,thick] (-0.3,-0.3)..controls (0.05,0).. (-0.3,0.3);
    \draw[color=gray,thick] (0.3,-0.3)..controls (-0.05,0).. (0.3,0.3);
  }%
}
\newcommand{\DSmooth}{%
  \KP{%
    \draw[color=gray,thick,->] (-0.3,0.3)..controls (0,-0.05).. (0.3,0.3);
    \draw[color=gray,thick,->] (-0.3,-0.3)..controls (0,0.05).. (0.3,-0.3);
  }%
}

\title{Geometric Topologies}
\author{Arkadiusz Naks}
\date{2023}

\begin{document}
\begin{titlepage}
  \begin{center}
    \makeatletter
    \vspace*{1cm}
    \Huge
    \textbf{\@title}

    \vspace{0.5cm}
    \Large
    Lecture notes from Geometric Topologies 2 module at Durham University

    \vspace{1.5cm}

    \textbf{\@author}

    \vfill

    \vspace{0.8cm}

    \small
    Based on my understanding of lectures and notes of Prof Anrew Jay Lobb, based on notes by Prof Dirk Shutz\\
    \@date{}
  \end{center}
\end{titlepage}

\tableofcontents
\newpage

\begin{section}{Important Definisions}
  A place for short and important definisions
  \textsc{Definision 1.10} (Achiral) \textit{A link which is isotopic to its mirror image is called \textbf{achiral}.}
\end{section}

\begin{section}{Basic Knot/Link Properties}

  \begin{subsection}{Tricolouring}
    Tricolouring can be used as one of the criteria to establish knot diagrams are different.
    This is invarient under all Reidemeister moves.

    Tricolouring rules:
    \begin{itemize}
      \item Each arc is assigned exacly one colour
      \item At least two colour are used
      \item At each crossing either 3 colours are used or only 1 colour is used
    \end{itemize}
    If all of these are met the diagram is called tricolourable.
  \end{subsection}

  \begin{subsection}{Writhe}
    Writhe of a knot diagram or an orientated link diagram D, mostly denoted as \(w(D)\),
    is a sum of all the positive crossings \(-\) the sum of all the negative crossings in D.
    Writhe of an orientated link diagram is dependent on the orientation of the link,
    and \emph{not invarient under R1 move}.
    Fromal definision is: \[w(D) = \sum_{x \in D} sign(x)\]
  \end{subsection}

\end{section}

\begin{section}{Knot/Link Polynomials}

\begin{subsection}{Bracket Polynomial}
Bracket Polynomial is a polynomial of an undirectd link in terms of A
The bracket polynomial is invarient under R0, R2 and R3.
Importantly bracket polynomial \emph{is not invarient under R1},
\[\left\langle D \right\rangle = -A^{3n} \left\langle D' \right\rangle\]
where D' is D after a R1 and \(n=1\) for positive R1 and \(n=-1\) for negative R1.

This are the basic rules that establis how the bracket polynomials work:
\begin{enumerate}
  \item \(\left\langle \Circle \right\rangle = 1\) where \(\Circle\) is the unknot
  \item\(\left\langle \Circle + D \right\rangle =- (A^{2} + A^{-2}) \left\langle D\right\rangle\) where D is any link diagram
  \item \(\left\langle \UCross \right\rangle = A \left\langle \RSmooth \right\rangle + A^{-1} \left\langle \LSmooth \right\rangle\)
\end{enumerate}

Bracket polynomials of some basic link diagrams:
\begin{itemize}
  \item Trefoil \(= A^{-7} - A^{-3} - A^{-5}\)
  \item Figure8 \(= A^{-8} - A^{-4} + 1 - A^{4} + A^{8}\)
  \item Hopflink \(= -A^{-4} - A^{4}\)
\end{itemize}
\end{subsection}

\begin{subsection}{X-Polynomial}
The X-polynomial, denoted as \(X(D)\) or \(X(D)[A]\),
is a development upon the bracket polynomial, now depending on the orientation of the link.
This polynomial is in terms of A.
This makes the X-polynomial \emph{invarient under all} Reidemeister moves.

This is achived through: \[X(D) = (-A){}^{-3w(D)} \left\langle D \right\rangle\]
where w is the writhe of D.

A knot is called achiral if \(X(D)[A] = X(D)[A^{-1}]\).
\end{subsection}

\begin{subsubsection}{Jones Polynomial}
The Jones polynomial, denoted as \(V_{D}(t)\), as is simply the X-polynomial with \(A = t^{-\frac{1}{4}}\), meaning it is definded as:
\[V_{D}(t) = X(D)[t^{-\frac{1}{4}}]\]
\end{subsubsection}

\begin{subsection}{Alexander-Conway Polynomial}
The Alexander-Conway polynomial, denoted \(\nabla_{K}(z)\)
is another polynomial to express a knot or link diagram. It also acts on orientated link diagrams.
There is no war to determin the achiral with this polynomial.

\begin{itemize}
  \item \(K_{+} = K\) is the original link contatint a positive crossing \(\DPCross\)
  \item \(K_{-}\) is the link after changing the crossing to \(\DNCross\)
  \item \(K_{0}\) is the link after changing the crossing to \(\DSmooth\)
\end{itemize}

Rules for computing the Alexandr-Conway polynomial
\begin{enumerate}
  \item \(\nabla_{U}(z) = 1\) where the U is the unknot
  \item \(\nabla_{K_{+}}(z) - \nabla_{K_{-}}(z) = z\nabla_{K_{0}}(z)\)
\end{enumerate}

Absolute polynomials of some basic link diagrams:
\begin{itemize}
  \item \(\nabla_{U_{2}} = 0\)
  \item Hopflink \(= z\)
  \item Trefoil \(z^{2} + 1\)
\end{itemize}
\end{subsection}

\begin{subsection}{Absolute Polynomial}
The absolute polynomial is the last polynomial tought on this course.
It is expressed as \(Q_{L}(x)\) and does not take into account the orientation of the link diagram.

\begin{itemize}
  \item \(L_{+}\) is the original containig \(\UCross\)
  \item \(L_{-}\) is the link after changing the crossing to \(\UOCross\)
  \item \(L_{0}\) is the link after changing the crossing to \(\RSmooth\)
  \item \(L_{\infty}\) is the link after changing the crossing to \(\LSmooth\)
\end{itemize}
\emph{Important note} both \(L_{+}\), \(L_{-}\) and  \(L_{0}\), \(L_{/infty}\) are interchangable!

Rules for compuing the absolute polynomial
\begin{enumerate}
  \item \(Q_{U}(x) = 1\) where U is the unknot
  \item \(Q_{L_{+}}(x) + Q_{L_{-}} = x(Q_{L_{0}} + Q_{L_{\infty}})\)
\end{enumerate}

Absolute polynomials of some basic link diagrams:
\begin{itemize}
  \item \(Q_{U_{2}} = 2x^{-1} - 1\)
  \item Hopflink \(= 2x + 1 + 2x^{-1}\)
  \item Trefoil \(= 2x^{2} + 2x - 3\)
\end{itemize}
\end{subsection}

\end{section}

\begin{section}{Surface Shit}

  \begin{subsection}{Surface}
    \textsc{Definision 2.1} (Surface) \textit{A \textbf{(parametric) surface S} in \(\mathbb{R}^{3}\) is a subset \(S \subset \mathbb{R}\)
      such that for every point \textbf{p} \(\in S\) there exists an injective, continously differentiable map \(\varphi: D^{2} \mapsto \mathbb{R}^{3}2\)
      satisfying the following:}
    \begin{enumerate}
      \item \textit{We have \(\varphi(\textbf{0}) = \textbf{p}\)} and \(\varphi(D^{2}) = S \cap D(\textbf{p}, r)\) for some \(r > 0\)
      \item \textit{The matrix of partial derivatives}
            \textit{\[ D_{\varphi}(a, b) = \begin{pmatrix}
              \frac{\partial \varphi_{1}}{\partial x_{1}}(a, b) & \frac{\partial \varphi_{1}}{\partial x_{2}}(a, b) \\
              \frac{\partial \varphi_{2}}{\partial x_{1}}(a, b) & \frac{\partial \varphi_{2}}{\partial x_{2}}(a, b) \\
              \frac{\partial \varphi_{3}}{\partial x_{1}}(a, b) & \frac{\partial \varphi_{3}}{\partial x_{2}}(a, b)
            \end{pmatrix}\]}
            \textit{has rank 2 for every \((a, b) \in D^{2}\)}
    \end{enumerate}
    \textit{Where \(\varphi(x, y) = (\varphi_{1}(x, y), \varphi_{2}(x, y), \varphi_{3}(x, y))\) and such a map is called a
    \textbf{parametrization around p} \(\in S\).}

    Insert actualy explenation of what a surface is here

    If S is a surface and \(\textbf{p} \in S\), then \(S \backslash{} \{{} \textbf{p} \}{}\) is also a surface.
  \end{subsection}

  \begin{subsection}{name2}
    some stuff here
  \end{subsection}

\end{section}

\end{document}
