\documentclass[12pt, letterpaper]{article}

\usepackage{graphicx}
\usepackage{parskip} % Disabling paragraph index as it does not fit maths
\usepackage{hyperref} % Usable menu and references
\usepackage{amssymb} % Used to show sets of sumbers, like the real numbers
\usepackage{amsmath} % Used for column vectors
\usepackage{xargs} % Used for multiple deafult command values

\graphicspath{{images}}

\newcommand{\R}{\mathbb{R}}
\newcommand{\C}{\mathbb{C}}
\newcommand{\Q}{\mathbb{Q}}
\newcommand{\Z}{\mathbb{Z}}

\title{Project Prereading}
\author{Arkadiusz Naks}
\date{2024}

\begin{document}

\tableofcontents
\newpage

\begin{section}{Representations}

  \begin{subsection}{Definision}

    For a field F and a group G, a \textbf{representation} of G over F is a pair
    \((\pi, V)\) where V is a vector space over F and \(\pi: G \to GL(V)\) a
    group homomorphism. It can also be said that \(\pi\) is a representation of
    G on V. The \textbf{dimension} of \((\pi, V)\) is the dimension of V but is
    often denoted as \(dim \; \pi\). \\
    This can be also thought about as the group G acting on the set V by
    \(g \cdot \textbf{v} := \pi(g) \textbf{v}\). As \(\pi\) is linear, the
    action is also linear. This can be reversly used to define a representation
    of G through an action of G on V.

    For a n dimentional representation, a basis can be chosen for V which
    \textit{identifies} V with \(F^{n}\). Then any invertibl linear map is the
    same thing as a \(n \times n\) matrix. Therefor any representation is
    just a homomorphism \(G \to GL_{n}(F)\). One dimensional representation of G
    is \(G \to F^{\times}\).

  \end{subsection}

  \begin{subsection}{Common Representations}

      Every group G has a trivial representation on every V where \(\pi\) sends
      everything in G to the identity.

      If the group G is cyclic of order n and generator g, the representation is
      completerly determined by V and \(\pi(g)\).

      Many representations of groups arise from geometry for groups such as
      \(\pi(r)\) mapping to a rotation matrix by \(\frac{2 \pi}{n}\) and
      \(\pi(s)\) mapping to a invertion of the second coordinate of \(v \in V\)
      for \(D_{n}\).

      Representations can also come from actions of groups on sets. For example,
      \((\pi, F^{n})\) of \(S_{n}\) s.t.\ \(\pi(\sigma)
      (\sum^{n} x_{i} \textbf{e}_{i}) = \sum^{n} x_{i} \textbf{e}_{\sigma(i)}\).
      Importantly for a column vector representation \[\pi(\sigma)
        \begin{pmatrix} x_{1} \\ \vdots \\ x_{n} \end{pmatrix} =
        \begin{pmatrix} x_{\sigma^{-1}(1)} \\ \vdots \\ x_{\sigma^{-1}(n)}
        \end{pmatrix}.
      \]
      More generally for a field F and a set X, define a free vecotr space
      \(F(X)\) by all formal sums \(\sum_{x \in X} z_{x} x\) where
      \(z_{x} \in F\) is non-zero for finitely many x. Then For a group acting
      on X, the \textbf{permutation representation} \((\pi, F(X))\) is defined
      as \[\pi(g)(\sum z_{x} x) := \sum z_{x} (g \cdot x) =
        \sum z_{g^{-1} \cdot x} x.\]

  \end{subsection}

  \begin{subsection}{Classification}

    \begin{subsubsection}{Homomorphism}

      For \((\rho, V)\) and \((\pi, W)\) representations of G, a G-homomorphism
      \(\phi: V \to W\) s.t.\
      \(\phi(\rho(g) \textbf{v}) = \pi(g) \phi(\textbf{v})\). \\
      Then \(Hom_{G}(V, W)\) is the vector space of all G-homomorphisms from V
      to W. These can also be called \textbf{intertwiners}. The inverse of every
      G isomorphism is also in the vector space. The collection of all
      representations isomorphic to \((\pi, V)\) are called its
      \textbf{isomorphic class}.

      For \(\phi \in Hom_{G}(V, W)\), \((\pi, ker(\phi))\) and
      \((\rho, im(\phi))\) are subrepresentations of V and W respctively.

      \begin{itemize}
        \item \(Hom_{G}(U \oplus V, W) \cong
              Hom_{G}(U, W) \oplus Hom_{G}(V, W)\)
        \item \(Hom_{G}(W, U \oplus V) \cong
              Hom_{G}(W, U) \oplus Hom_{G}(W, V)\)
      \end{itemize}

    \end{subsubsection}

    \begin{subsubsection}{Subrepresentations}

      For a representation \((\pi, V)\), a \textbf{subrepresentation} is
      \((\pi_{|W}, W)\) where \(W \subset V\) and \(\pi_{|W}\) is just \(\pi\) but
      only on W. The subset has to be closed; \(\pi(g) \textbf{w} \in W\). \\
      As one dimensional vector spaces do not have non-trivial subspaces, one
      dimensional representations have no subrepresentations.

    \end{subsubsection}

    \begin{subsubsection}{Irreducibility}

      If a non-zero representation has no non-trivial subrepresentations, it is
      called \textbf{irreducible}.

      The goal of representation theory is to classify all irreducable
      representations for any group G. This is achived by a list with L
      \begin{itemize}
        \item Any irreducable representation of G is isomorphic to some entry in
              L
        \item All entries in L are non isomorphic to each other
      \end{itemize}

      For a group G, any irreducable representation \(\pi\) has
      \(dim \; \pi \leq |G|\), meaining its finite dimentional.

    \end{subsubsection}

    \begin{subsubsection}{Schur's Lemma}

      If \((\pi, V)\) is irreducible complex representation of
      G, \[Hom_{G}(V) = \C I = \{cI | c \in \C\} (methinks).\] This means that
      for two irreducible representations \((\pi, V)\) and \((\rho, W)\) of
      G, \[dim Hom_{G}(V, W) =
        \begin{cases}
          1 & (\pi, V) \cong (\rho, W) \\
          0 & otherwise.
        \end{cases}\]

      This implies that every finite-dimensional irreduceble complex
      representation of any \textbf{abelian} group G is \textbf{one
        dimensional}. A homomorphism \(\chi: G \to \C^{\times}\) is often called
      a \textbf{character} (this notation will clash later in the course so call
      it \textbf{character\(_{hom}\)}). For any abelian G, let \(\hat{G}\)
      denote the character\(_{hom}\) group or the \textbf{dual group}. This
      group is isomorphic to the original group.

      In fact every center Z of a group G has some \textbf{central
        character\(_{hom}\)} \(\chi: Z \to \C^{\times}\) s.t.\
      \(\pi(z) \textbf{v} = \chi(z) \textbf{v}\).

      A tighter upper bound can be imposed on the size of irreducable
      representations of G, namely for any abelian subgroup A of G
      \(dim V \leq [G : A]\).

    \end{subsubsection}

    \begin{subsubsection}{Operations}

      A representation \((\pi, V)\) is \textbf{reducable} or
      \textbf{decomposable} if it has two non trivial subrepresentations
      \((\pi, U)\) and \((\pi, W)\) s.t.\ \(V = U \oplus W\). This means
      \((\pi, V) \cong (\pi, U) \oplus (\pi, W)\). A representation can be not
      reducible nor irreducible. However this is only possible for infinite
      groups. Any representation \((\pi, V)\) of a \textbf{finite} group that
      has a subrepresentation \((\pi, W)\), also has a subrepresentation
      \((\pi, W')\) s.t.\ \((\pi, V) = (\pi, W) \oplus (\pi, W')\). \\
      In fact all \textbf{unitary} representations are reducable with
      \(W' = W^{\bot}\); for
      \(W^{\bot} = \{v \in V | \langle v, w \rangle = 0\}\). If W is a
      subrepresentation, so is \(W^{\bot}\).

      A representation \((\pi, V)\) with V a vector space is \textbf{unitary}
      if \[\langle \pi(g) \textbf{v}, \pi(g) \textbf{u} \rangle =
        \langle \textbf{v}, \textbf{u} \rangle.\] This is equivelant to
      \(\langle \pi(g) \textbf{v}, \textbf{u} \rangle =
      \langle \textbf{v}, \pi(g){}^{*} \textbf{u} \rangle\) where
      \(\pi(g){}^{*} = \pi(g){}^{-1}\). The inner product is called
      \textbf{G-invariant}. Every representation has a \textbf{G-invariant}
      inner product.

    \end{subsubsection}

    \begin{subsubsection}{Decomposition}

      Any representation \[(\pi, V) =
        (\pi, W_{1}) \oplus \cdots \oplus (\pi, W_{m})\] where all \(W_{i}\) are
      irreducable. Morever, the number of each isomorphism class of
      \((\pi, W_{n})\) appears is unique and independent of the exact choice of
      decomposition. In fact each isomorphism class \((\rho, U)\) appears
      \(dim(Hom_{G}(V, U))\) times.

    \end{subsubsection}

  \end{subsection}

  \begin{subsection}{The Group Algebra}

    For a finite group G, denfine the free vector space \(\C(G)\) as all formal
    sums \(\sum_{g \in G} z_{g} g\) where \(z_{g} \in \C\). Define vector
    multiplication as \[(\sum z_{g} g)(\sum w_{h} h) := \sum z_{g}w_{h} gh =
      \sum_{g}(\sum_{h} z_{gh^{-1}} w_{h}) g.\] The vector space together with
    this multiplication is called the \textbf{group algebra} \(\C[G]\) of G.
    This multiplication can also be executed by simple multiplication similar to
    regular integers.

    An \textbf{algebra} is a vector space V over a field F with a multiplication
    \(V \times V \to V\) s.t.\
    \begin{itemize}
      \item \(\textbf{v} (\textbf{u} + \textbf{w}) =
            \textbf{vu}\ + \textbf{vw}\)
      \item \((\textbf{u} + \textbf{w}) \textbf{v}=
            \textbf{uv}\ + \textbf{wv}\)
      \item \((\alpha \textbf{v})(\beta \textbf{u}) =
            (\alpha \beta) \textbf{vu}\)
    \end{itemize}

    G is the basis of \(\C[G]\) (formally \(1g \; \forall g \in G\)) so
    \(dim \C[G] = |G|\). For any representation \((\pi, V)\)
    \(Hom_{G}(\C[G], V) \cong V\). \\
    For \(Irr(g)\) a set of non-isomorphic irreducable representation
    \((\pi, W_{\pi})\)
    \[(\lambda, \C[G]) = \bigoplus_{\pi \in Irr(G)} (\pi, W_{\pi}){}^{dim \pi}.\]
    This implies that \(\sum_{\pi \in Irr(G)} dim(\pi){}^{2} = |G|\). This is
    very useful to check have all irreducable representations been found and
    verify that they are non-isomorphic.

  \end{subsection}

\end{section}

\begin{section}{Characters}

  Note: \emph{These are different from the characters mentioned earlier} \\
  In this section G is finite and V is finite-dimensional over \(\C\).

  \begin{subsection}{Definision}

    For \((\pi, V)\) a representation of G, \(\chi_{\pi}: G \to \C\) is defined
    as \[\chi_{\pi}(g) = tr(\pi(g)).\] \emph{Trace of a matrix is equall to the
      sume of its eigenvalues} \\
    Isomorphic representations have the same characters and characters
    completely determin representations.

    Some properites of character
    \begin{itemize}
      \item \(\chi_{\pi}(gh) \neq \chi_{\pi}(g) \chi_{\pi}(h)\)
      \item \(\chi_{\pi}(e) = dim V\)
      \item \(\chi_{\pi}(gh) = \chi_{\pi}(hg)\)
      \item \(\chi_{\pi \oplus \rho} = \chi_{\pi} + \chi_{\rho}\)
      \item \(\chi_{\pi}(g^{-1}) = \bar{\chi_{\pi}(g)}\)
    \end{itemize}

    \begin{subsubsection}{Common Examples}

      If the representation is one dimentioal, it is its own character.

      For the permutation representation of \(S_{n}\), \[\chi_{\pi}(\sigma) =
        |\{j \in \{1, \dots, n\} | \sigma(j) = j\}|.\]

      For the regular representation \[\chi_{\pi}(g) =
        \begin{cases}
          |G| & g = e \\
                 0
        \end{cases}
      \]

    \end{subsubsection}

  \end{subsection}

  \begin{subsection}{Irreducibles}

    The character of a representation is constant for all elements of any
    \textbf{conjugacy class} of G. Also every character can be writen as an
    integral combination of irreducibles.

    The information about irreducible characters of G is often expressed as a
    table where the top row is a list of all conjugacy classes, the second is
    the size of each class and the following are all characters corresponding to
    the particular irreducable representation. \\
    The character part of tat table is always square, meaning the number of
    irreducible representations is equal to the number of conjugacy classes.

  \end{subsection}

  \begin{subsection}{Class Function}

    A class function \(G \to \C\) is a function that is constant on conjugacy
    classes. The set of all such functions of G is denoted \(CF(G)\). In that,
    this set is a vector space with dimension equal to the number of conjugacy
    classes. The inner product is defined as \[\langle f_{1}, f_{2} \rangle_{G}
      := \frac{1}{|G|} \sum f_{1}(g) \bar{f_{2}(g)} = \frac{1}{|G|}
      \sum_{\mathcal{C} \in conj(G)} |\mathcal{C}| f_{1}(\mathcal{C})
      \bar{f_{2}(\mathcal{C})}\] where \(cong(G)\) denotes the set of all
    conjugacy classes of G. \\
    \(\{\chi_{\rho}\}{}_{\rho \in irr(G)}\) is an
    orthogonal basis of \(CF(G)\) with respect to the inner product.

    Let \([\mathcal{C}] \in \C^{|irr(G)|}\) be the column of the character
    table corresponding to \(\mathcal{C}\).
    Then \[[\mathcal{C}] \cdot [\mathcal{D}] =
      \begin{cases}
        \frac{|G|}{|\mathcal{C}|} & \mathcal{C} = \mathcal{D} \\
        0
      \end{cases}\]

  \end{subsection}

  \begin{subsection}{Orhtogonality}

    For reps \((\pi, V)\) and \((\rho, W)\), define
    \((C^{\rho}_{\pi}, Hom(V, W))\) as another representation of G
    s.t.\ \[c^{\rho}_{\pi}(g)(T) := \rho(g)T\pi(g^{-1}).\]

    \(Hom_{G}(V, W) = \{T \in Hom(V, W) | c^{\rho}_{\pi}(g)T = T,
    \forall g \in G\}\) where \(Hom(V, W)\) are vector space homomorphisms and
    \(Hom_{G}(V, W)\) are representation homomorphisms.
    Also \[\chi_{c^{\rho}_{\pi}} = \chi_{\rho} \bar{\chi_{\pi}}.\]

    Define \(V^{G}\) to be the set of fixed points of the representation
    \((\pi, V)\). Then the subrepresentation \((\pi, V^{G}) = (Id, V^{G})\).
    Then \(dim(V^{G}) = \frac{1}{|G|} \sum \chi_{\pi}(g)\).

    \[\langle \chi_{\rho}, \chi_{\pi} \rangle = dim Hom_{G}(V, W).\]

    As mentioned above, two reps are isomorphic iff they have the same
    character. Also a rep \((\pi, V)\) is irred iff \[||\chi_{\pi}||^{2}_{G} =
    \langle \chi_{\pi}, \chi_{\pi} \rangle_{G} = 1.\]

  \end{subsection}

  \begin{subsection}{Universal Projections}

    \(\frac{1}{|G|} \sum \pi(g)\) acts on V as the \textbf{orthogonal
      projection} onto \(V^{G}\).

    The space of \(\C\) valued functions on G is isomorphic to \(\C[G]\), with a
    possible isomorphism \(f \to \frac{1}{|G|} \sum \bar{f(g)} g\). Then
    define \[\pi(f) := \frac{1}{|G|} \sum \bar{f(g)} \pi(g).\] if
    \(f \in CF(G)\) then \(\pi(f) \in Hom_{G}(V)\). \\
    For an irreducible rep \((\rho, W)\), \(\rho(f) = \frac{1}{dim(W)}
    \langle \chi_{\rho}, f \rangle_{G} I\).

    For two irreducible reps, \[\pi(\chi_{\rho}) =
      \begin{cases}
        \frac{1}{dim(V)} I & (\pi, V) \cong (\rho, W) \\
        0
      \end{cases}\]

    The subrep \(\pi, V_{\rho}\) is called the \(\rho\)-isotopic coponent of
    \(\pi\). For \((\pi, V) = \bigoplus (\pi, W_{i})\) an irred decomposition, for
    each
    \(\rho \in Irr(G)\) \[(\pi, V_{\rho}) =
      \bigoplus_{i, \; \pi|_{W_{i}} \cong \rho} (\pi, W_{i}).\]
    Then \(\rho\)-isotopic projector is the \(dim(\rho) \pi(\chi_{\rho})\) and
    it is the projection onto \(V_{\rho}\). Both of these are independent of the
    choice of decomposition.

    \end{subsection}

\end{section}

\begin{section}{Constructing Representations}

  \begin{subsection}{Linear Algebra}

    \begin{subsubsection}{Dual Representation}

      Given a vecotr space \(V\), its dual is the vector space consisting of all
      linear functions \(V \to \C\), \(V^{*} = Hom(V, \C)\). Define
      \((\pi^{*}, V^{*})\) of G as \[[\pi^{*}(g) \lambda](v) :=
        \lambda (\pi(g^{-1})v)\] for an existing representation \((\pi, V)\).
      Then \(\chi_{\pi^{*}} = \bar{\chi_{\pi}}\).

    \end{subsubsection}

    \begin{subsubsection}{Quotient Representation}

      For a representation \((\pi, V)\) and a subrepresentation
      \((\pi|_{W}, W)\), \((\hat{\pi}, \frac{V}{W})\) is also a representation
      of G with \(\hat{\pi}(g)(v) := \pi(g)v\) modulo W. It is critical that
      \(\pi(g)W = W\) (W is a subrepresentation). \\
      If \((\pi, V) = (\pi, W), \oplus (\pi, W')\) then
      \((\hat{\pi}, \frac{V}{W}) \cong (\pi, W')\). This means
      \(\chi_{\hat{\pi}} = \chi_{\pi} - \chi_{\pi|_{W}}\).

    \end{subsubsection}

    \begin{subsubsection}{Tensor Product}

      Define two representations \((\pi, V)\) and \((\rho, W)\). Firstly
      consider the permutation representation of G on \(\C(V \times W)\). Denote
      it as \(\pi \otimes \rho\). Each \(a \in \C(V \times W)\) is of the
      form \[\sum_{v, w \in V \times W} z_{v, w} (v, w)\] for
      \(z_{v, w} \in \C\) with all but finitely many of them non zero. Note that
      this is a infinite dimensional vector space. Then define the action
      as \[[\pi \otimes \rho](g) = \sum z_{(v, w)} (\pi(g)v, \rho(g)w).\]

      This vector space is not good for a representation as
      \((v_{1} + v_{2}, w) \neq (v_{1}, w) + (v_{2}, w)\). To fix this define a
      subset \[U = span\{(\alpha v_{1} + \beta v_{2}, \lambda w_{1} + \mu w_{2})
        - \alpha \lambda (v_{1}, w_{1}) - \alpha \mu (v_{1}, w_{2}) - \beta
        \lambda (v_{2}, w_{1}) - \beta \mu (v_{2}, w_{2})\}.\] Then define
      \[(\pi \otimes \rho, V \otimes W) := (\pi \otimes \rho, \C(V, W) / U).\]
      (\((\pi \otimes \rho, U)\) is also a representation).

      \(\{v_{i} \otimes w_{j} \}\) for all \(v_{i}\) and \(w_{j}\) in the basis
      of V and W is a basis of \(V \otimes W\). This means \(dim(V \otimes W) =
      dim(V)dim(W)\). An example is \(\C^{n} \times \C^{m} \cong \C^{mn}\). An
      element of the form \(v \otimes w\) is called a \textbf{pure tensor}.
      \(\chi_{\pi \otimes rho} = \chi_{\pi} \chi_{rho}\). This means
      \((c^{\rho}_{\pi}, Hom(V, W)) \cong (\rho, W) \times (\pi^{*}, V^{*})\).

    \end{subsubsection}

    \begin{subsubsection}{Symmetric Product}

      Take \(U_{1} = span \{v \otimes w - w \otimes v\} \subseteq V \otimes V\)
      and \(Sym^{2}(V) := (V \otimes V) / U_{1}\). The resulting vector space is
      called the (second) \textbf{symmetric} product of V. This makes \(v_{1}
      \otimes v_{2} = v_{2} \otimes v_{1}\). Then elements can be expressed as
      \(v_{1}v_{2}\). \(dim(Sym^{2}(V)) = \frac{n(n + 1)}{2}\).

      \(\chi_{Sym^{2}(\pi)}(g) = \frac{1}{2} (\chi_{\pi}(g){}^{2} +
      \chi_{\pi}(g^{2}))\)

    \end{subsubsection}

    \begin{subsubsection}{Alternating Product}

      Similarly to the symetric case, define \(U_{2} = span \{v_{1} \otimes
      v_{2} + v_{2} \otimes v_{1}\}\). Then
      \(\bigwedge^{2} V = V \otimes V / U_{2}\) is the (second)
      \textbf{alternating} product of V enforcing
      \(v_{1} \otimes v_{2} = -v_{2} \otimes v_{1}\).
      \(dim(\bigwedge^{2}(V)) = \frac{n(n - 1)}{2}\).

      \(\chi_{\pi \wedge \pi}(g) = \frac{1}{2} (\chi_{\pi}(g){}^{2} -
      \chi_{\pi}(g^{2}))\)

    \end{subsubsection}

    \begin{subsubsection}{Symmetric and Alternating Decomposition}

      \[(\pi \otimes \pi, V \otimes V) = (Sym^{2}(\pi), Sym^{2}(V)) \oplus
        (\pi \wedge \pi, \bigwedge^{2}(V)).\]

    \end{subsubsection}

  \end{subsection}

  \begin{subsection}{Group Theory}

    \begin{subsubsection}{Inflation}

      For N a normal subgoup of G s.t.\ \(g_{1} \cdot g_{2} = g_{1}g_{2}\)
      \(\forall g \in \tilda{G} = G / N\). Then the \textbf{inflation} of
      \(\pi\) to G is denoted \((\tilda{\pi}, V)\) and defined as
      \(\tilda{\pi}(g) := \pi(gN)\). This is also called the \textbf{lift} of
      \(\tilda{G}\) to \(G\).

      A rep \(\pi\) of G may be expressed as the inflation of a rep of
      \(\tilda{G}\) iff \(N \subseteq \ker(\pi)\).

\end{section}

\end{document}
