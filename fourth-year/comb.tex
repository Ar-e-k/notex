\documentclass[12pt, letterpaper]{article}

\usepackage{graphicx}
\usepackage{parskip} % Disabling paragraph index as it does not fit maths
\usepackage{hyperref} % Usable menu and references
\usepackage{amssymb} % Used to show sets of sumbers, like the real numbers
\usepackage{amsmath} % Used for column vectors
\usepackage{xargs} % Used for multiple deafult command values

\graphicspath{{images}}

\newcommand{\R}{\mathbb{R}}
\newcommand{\C}{\mathbb{C}}
\newcommand{\Q}{\mathbb{Q}}
\newcommand{\Z}{\mathbb{Z}}

\title{Project Prereading}
\author{Arkadiusz Naks}
\date{2024}

\begin{document}

\tableofcontents
\newpage

\begin{section}{Introduction}

  \begin{subsection}{Catalan Numbers}

    A \textbf{Ballot} sequence is a sequence \((\epsilon_{1}, \dots,
    \epsilon_{2n})\) where \(\epsilon_{i} = \pm 1\) and
    \begin{itemize}
      \item \(\sum^{2n}_{i} \epsilon_{i} = 0\)
      \item \(\sum^{k}_{i} \epsilon \geq 0\) for \(k \leq 2n\)
    \end{itemize}
    Clearly every such sequence starts with \(1\) and ends in \(-1\).

    The n-th \textbf{Catalan} number denoted \(C_{n}\) is the number of such
    sequences of lenght \(2n\). This is not the only definision, equivalently
    \begin{itemize}
      \item The number of bracket arrangements in a non-associative product of
            \(n + 1\) variables (non trivial)
      \item Number of triangulations of a convex \((n + 2)\)-gon on a place
    \end{itemize}

    \begin{subsubsection}{Dyck Path}

      In a \(n \times n\) square, a \textbf{Dyck} path is a path going from one
      corner to the opposite one, for example \((0, n) \to (n, 0)\) while staing
      above (or on) the main diagonal of the square. \\
      They can be related to Ballot sequences easly by taking \(\epsilon = 1\)
      to mean a move to the right and \(\epsilon = -1\) to be a move down (for
      the choice of corners above). \\
      This can also taken as the triangular top right half of the square only
      with the diagonal acting as the x-axis.

      Both of these models of Dyck paths are \(\Z^{2}\) \textbf{lattice} paths.

      Define a peak of a path as a local maximum nd valley as a local minimum.
      Then there is one more peak than valley.

    \end{subsubsection}

    \begin{subsubsection}{Catalan Explicitly}

      \[C_{n} = \frac{1}{n + 1} {{2n}\choose{n}}\].
      Equivelantely, the Catalan numbers can be defined recurrently
      as \[C_{n} = \sum^{n}_{k=1} C_{k-1} C_{n-k}.\] For this purpouse,
      \(C_{0}\) is defined as \(1\).

      For a non-negative integer sequence \((a_{n})\), a \textbf{generating
        function} is defined as \(A(x) = \sum^{\infty}_{k=0} a_{k} x^{k}\). \\
      For the series \((C_{n})\), the generating
      function \[C(x) = \frac{1 - \sqrt{1 - 4x}}{2x},\]
      which is derived from the fact that \[xC(x)^{2} - C(x) + 1 = 0.\]

    \end{subsubsection}

  \end{subsection}

  \begin{subsection}{Young Diagram}

    \begin{subsubsection}{Definision}

      A partition on n integers is a sequence of integers
      \(\lambda = (\lambda_{1}, \dots, \lambda_{k})\) s.t.\
      \(\sum^{k} \lambda_{i} = n\) and \(\lambda_{i} \geq \lambda_{j} > 0\) for
      \(i > j\). These can be represented as a diagram of squares with
      \(\lambda_{i}\) squares in the ith row, called a Young diagram. \\
      Then a \textbf{standard Young tableau} (SYT) is a filling of a Young
      diagram by numbers \(1, \dots, n\) with each number appearing once and the
      entries increasing both down the columns and right in rows. Then let
      \(f_{\lambda}\) denote the number of possible SYTs

    \end{subsubsection}

    \begin{subsubsection}{Calculating SYT}

      The \textbf{hook} of the \((i, j)\) square is \(1 + \) all boxes below
      \(+\) all boxes to the right. This is denoted as \(h(i, j)\). \\
      Then denote \(H(\lambda) = \Pi_{(i, j) \in \lambda} h(i, j)\). It is
      useful as \(f_{\lambda} = \frac{n!}{H(\lambda)}\). \\
      \(f_{0}\) is defined to be \(1\). Then \(f_{\lambda} =
      \sum f_{\lambda - c}\) where c is all the bottom right corners.

      Define weight as \(wt(p, q) = \frac{1}{h(p, q) - 1}\).

    \end{subsubsection}

  \end{subsection}

  \begin{subsection}{Set Partitions}

    \begin{subsubsection}{Definision}
      
      Denote \([n]\) the set of integers \(1, \dots, n\), then a set partition
      of \([n]\) a subdivision itno disjoint union of non-empty subsets. This is
      often represented as dots representing numbers and lines connecting the
      dots iff they are the closest numbers in its subpartition. The number of
      set partitions is called the \textbf{Bell number} \(B(n)\). The more
      rigorous case is partitioning the numbers into k partitions, counted by
      the \textbf{Stirling number of second kind}, \(S(n, k)\). Clearly
      \(B(n) = S(n ,k)\). \\
      The exponential generating function for Bells numbers is
      \(B(x) = \exp(e^{x} - 1)\).

    \end{subsubsection}

    \begin{subsubsection}{Special Partitions}

      A set partition can \textbf{non-crossing} or \textbf{non-nesting}. The
      number of non-nesting set partitions is equal to \(C_{n}\). The number of
      non-nesting partitions with k blocks is equal to \(N(n, n - (k - 1))\)
      where \(N(n, k)\) is the \textbf{Narayana number}, the number of Dyck
      paths with k peaks.

    \end{subsubsection}

    \begin{subsubsection}{Generating Functions}

      An \textbf{exponential} generating function for a sequence \((a_{n})\) if
      a formal power series \(\sum^{\infty} a_{n} \frac{x^{n}}{n!}\). This is
      usually used for labelled objects as opposed to regular generating
      functions being used for unlabelled ones.

      For \(c_{n}\) counting some objects on n labelled nodes and \(c(x)\) its
      exponential genrating function, define \(d_{n}\) as the number of
      collections of c-objects on \(n_{1}, \dots, n_{k}\) elements s.t.\
      \(\sum^{k}_{i = 1} n_{i}\).  Then \(d(x) = \exp(c(x))\).

    \end{subsubsection}

    \begin{subsubsection}{Integer Partition}

      For a partition \(\lambda = (\lambda_{1}, \dots, \lambda_{k})\), it can
      also be writen as \(\lambda = 1^{m_{1}} 2^{m_{2}} \dots\) where \(m_{i}\)
      is the number of \(\lambda_{j} = i\).

    \end{subsubsection}

  \end{subsection}

\end{section}

\begin{section}{Permutations}

  One line notation is the same as two line notation except for the top line is
  ignored.

  For a permutation \(w \in S_{n}\), define an \textbf{inversion} as a pair
  \((i < j)\) s.t.\ \(w_{i} > w_{j}\) where w writen in one line notation. Also
  define \textbf{descent} as \(i < n\) s.t.\ \(w_{i + 1} < w_{i}\). This is
  essentially an inversion next to each other.

  \begin{subsection}{Statistics}

    A statisctic on a permutaion is a function \(\mu: S_{n} \to \Z_{\geq 0}\).
    Its generating function is \(f_{\mu}(x) = \sum_{w \in S_{n}} x^{\mu(w)}\).
    If two generating functions are equivelant, the statistics are said to be
    \textbf{equidistributed}.

    \begin{subsubsection}{Basic Examples}

      Define \(inv(w)\) was the number of inversions in w, \(des(w)\) as the
      number of descents and \(cyc(w)\) as the number of cycles. Then any
      statistic is called \textbf{Mahonian} if it is equidistributed to inv,
      similarly with \textbf{Eulerian} and des.

      \[F_{inv}(x) = (1 + x)(1 + x + x^{2}) \cdots (1 + x + \dots + x^{n - 1}) =
        \frac{\prod^{n} (1 - x^{k})}{(1-x)^{n}}.\]

    \end{subsubsection}

    \begin{subsubsection}{Examples}

      Define the \textbf{Major index} as
      \(maj(w) = \sum _{k \; a \; descent} k\)
      (where k is the index). This statistic is Mahonian.

      Define a \textbf{record} as an element s.t.\ \(w_{i}\) is a record if
      \(w_{i} > w_{j}\) \(\forall j < i\). The number of records is a statistic
      equidistributed to cyc and denoted \(rec\).

      Define an \textbf{exceedance} is an index i s.t.\ \(i < w_{i}\). Then the
      number of exceedances \(exc\) is an Eulerian statistic.

    \end{subsubsection}

  \end{subsection}

\end{section}

\end{document}
